\newcommand{\taosao}[1]{%
	\ifcase#1\relax%
	\or
	\faStar~\faStarO~\faStarO~\faStarO~\faStarO
	\or
	\faStar~\faStar~\faStarO~\faStarO~\faStarO
	\or
	\faStar~\faStar~\faStar~\faStarO~\faStarO
	\or
	\faStar~\faStar~\faStar~\faStar~\faStarO
	\else
	\faStar~\faStar~\faStar~\faStar~\faStar
	\fi
}
\renewenvironment{center}
{\parskip=3pt\par\nopagebreak\centering}
{\par\noindent\ignorespacesafterend}

\renewcommand*\circled[1]{\tikz[baseline=(char.base)]{
		\node[shape=circle,inner sep=1pt,font=\footnotesize\sffamily,fill=\mycolor] (char) {#1};}}
	
\renewcommand*\circEX[1]{\tikz[baseline=(char.base)]{\node[shape=circle,draw=cyan,inner sep=1pt,fill=cyan!15,font=\footnotesize] (char) {\color{orange} #1};}}

\renewcommand*\squareEX[1]{\tikz[baseline=(char.base)]{\node[shape=rectangle,draw=red,inner sep=2.5pt,fill=yellow!50] (char) {\color{red} #1};}}

\newcommand\circlenumH[2][\mycolor]{\tikz[baseline=(char.base)]
	{\node [shape=circle,inner sep=0.5pt,draw=#1!90!black,fill=#1!20,double,
	font=\bfseries\large,minimum size=15pt,outer sep=0pt,anchor=base](char) {#2};
	}}
		   
\newcommand\circleTrue[2][\mycolor]{\tikz[baseline=(char.base)]{
		\node[shape=circle,draw=#1,inner sep=1.5pt,fill=#1!10,font=\sffamily] (char) {\sffamily\color{#1}{\textbf{#2}}};}}
	
\newcommand\ntdshape[1]{\tikz[baseline=(char.base)]{\node[shape=circle,inner sep=1.5pt,fill=cyan!30!white,minimum size=15pt] (char) {%
			#1};}}
		
\newcommand{\danhsoovuong}[1]{
	\tikz[anchor=base, baseline] {
		\node[inner sep=5pt] (MUC) {\bfseries#1};
		\path[rounded corners=2pt,fill=\mycolor,
		preaction={transform canvas={shift={(-45:2.5pt)}},left color=\mycolor!75!black,right color=\mycolor!50!black}] ($(MUC.north west)+(1.5pt,-1.5pt)$) -- ($(MUC.north east)+(1mm,0mm)+(1.5pt,-1.5pt)$) -- ($(MUC.south east)+(1.5pt,-1.5pt)$) -- ($(MUC.south west)+(-1mm,0)+(1.5pt,-1.5pt)$) -- cycle;
		\path ([shift={(2.05pt,-6.05pt)}]MUC) node[inner sep=5pt]{\color{gray}\slshape\bfseries#1};
		\path ([shift={(2pt,-6pt)}]MUC) node[inner sep=5pt]{\color{white}\slshape\bfseries#1};
	}%
}

\newcommand{\khungvuong}[1]{
	\tikz[anchor=base, baseline] {
		\node[inner sep=5pt] (MUC) {\bfseries#1};
		\path[rounded corners=2pt,fill=\mycolor!90,
		preaction={transform canvas={shift={(-45:2.5pt)}},left color=\mycolor!80!black,right color=\mycolor!80!black}] ($(MUC.north west)+(1.5pt,-1.5pt)$) -- ($(MUC.north east)+(1mm,0mm)+(1.5pt,-1.5pt)$) -- ($(MUC.south east)+(1.5pt,-1.5pt)$) -- ($(MUC.south west)+(-1mm,0)+(1.5pt,-1.5pt)$) -- cycle;
		\path ([shift={(2pt,-6pt)}]MUC) node[inner sep=5pt]{\color{white}\slshape\bfseries#1};
	}%
}


%%%=====================Tìm hiểu gói tạo dòng kẻ===============%%%
%\taodongke{5}
%\taodongke[1.5]{5}
%\taodongke[1.5][\hrule]{5}
%\taodongke[1.5][\hrule][red]{5}
%\taodongke[1.5][][red]{5}
%\taodongke[][\hrule][red]{5}
%\taodongke[][][red]{số dòng}
%%%tùy chọn bỏ trống hai%\hfill %\dotfill%\hrule
\usepackage{multido} % gói thêm dòng kẻ
\NewDocumentCommand{\taodongke}{O{}O{}O{black}m}{%
	\ifblank{#1}{\def\tuychonone{1.12}}{\def\tuychonone{#1}}
	\ifblank{#2}{\def\tuychontwo{\dotfill}}{\def\tuychontwo{#2}}
	\noindent%
	\pgfmathparse{int(#4-1)}
	\ifnum\pgfmathresult>1
	\multido{}{\pgfmathresult}{\noindent\rule{0pt}{\tuychonone\baselineskip}%
		{\color{#3}\mbox{}\tuychontwo\endgraf}}%
	\fi%
	{\noindent\rule{0pt}{\tuychonone\baselineskip}\color{#3}\mbox{}\tuychontwo\endgraf}}%
\newcommand{\daucham}[1]{\leavevmode\hbox to #1{\cleaders\hbox to .35em{\raisebox{-1pt}.}\rule{0pt}{\baselineskip}\hfil}\ignorespaces}

%%%%==========Chuyển nội dung thành số dòng kẻ==================%%%

\newcommand{\doithanhdongke}[1]{%
	\setbox0=\vbox{\hbox{
			\noindent\begin{minipage}{\linewidth}%
				#1 
			\end{minipage}
	}}
	\linepar=\ht0
	\pgfmathparse{int((\linepar-\fboxsep)/(\lineheight)+1)}
	\taodongke{\pgfmathresult}
}





\def\phangiac(#1,#2,#3)(#4){
	\path 
	($ (#2)!1cm!(#1) $)  coordinate (#2#1)
	($ (#2)!1cm!(#3) $)  coordinate (#2#3)
	($(#2#1)!0.5!(#2#3)$) coordinate (#2t)
	(intersection of #1--#3 and #2--#2t) coordinate (#4);
}

\def\khmotgach(#1,#2){%
	\path (#1)--($ (#1)!0.5!(#2) $)--([turn]90:2pt) coordinate (AtB)
	--([turn]180:4pt) coordinate (BtA);
	\draw (AtB)--(BtA);
}

\def\khhaigach(#1,#2){%
	\path (#1)--($ (#1)!0.495!(#2) $)--([turn]90:2pt) coordinate (AtB)
	--([turn]180:4pt) coordinate (BtA);
	\draw (AtB)--(BtA);
	\path (#1)--($ (#1)!0.505!(#2) $)--([turn]90:2pt) coordinate (AtB)
	--([turn]180:4pt) coordinate (BtA);
	\draw (AtB)--(BtA);
}

\def\khdaux(#1,#2){%
	\path (#1)--($ (#1)!0.5!(#2) $)--([turn]60:2pt) coordinate (AtB)
	--([turn]240:4pt) coordinate (BtA);
	\draw (AtB)--(BtA);
	\path (#1)--($ (#1)!0.5!(#2) $)--([turn]120:2pt) coordinate (AtB)
	--([turn]300:4pt) coordinate (BtA);
	\draw (AtB)--(BtA);
}

\def\kyhieudolon(#1,#2)(#3){\path (#1)--(#2)--([turn]270:9pt) coordinate (X);
	\path ($ (X)+(#1)-(#2) $) coordinate (Xt);
	\draw[dotted] (#1)--(Xt) (#2)--(X);
	\draw[|<->|,>=stealth] (Xt)--(X) node[fill=white,inner sep=1pt,%
	midway,sloped]{#3};
}
\newcommand{\indam}[2][\maunhan]{{\bfseries\sffamily\color{#1}{#2}}}

\NewDocumentCommand{\NTDtext}{O{}O{}O{black}m}{%
	\ifblank{#1}{\def\tuychonone{\normalsize}}{\def\tuychonone{#1}}
	\ifblank{#2}{\def\tuychontwo{\mdseries}}{\def\tuychontwo{#2}}
	{\par\noindent\tuychonone\tuychontwo\color{#3}{#4}}
}
%%% Tùy chọn 1 : font chữ có thể bỏ trống
%%%\tùy chọn 2 :nhãn có thể bỏ trống
%%%\tùy chọn 3: màu
\newcounter{gsnd}
\renewcommand{\thegsnd}{\arabic{gsnd}}
\NewDocumentCommand{\GSND}{O{}O{}O{\maunhan}m}{%
	\stepcounter{gsnd}
	\ifblank{#1}{\def\tuychonone{\bfseries\fontfamily{qag}\selectfont}}{\def\tuychonone{#1}}
	\ifblank{#2}{\def\tuychontwo{\faAdjust\;}}{\def\tuychontwo{#2\;}}
	{\par\noindent\tuychonone\color{#3}{\tuychontwo#4}}
}

\NewDocumentCommand{\MyNode}{O{}O{}O{\maunhan}m}{%
	\ifblank{#1}{\def\optonce{\mauphu}}{\def\optonce{#1}}
	\ifblank{#2}{\def\opttwo{4pt}}{\def\opttwo{#2}}
	\tikz{%
		\node[%
		draw = \optonce, fill= \optonce!20, rounded corners = \opttwo,font = \color{#3}\Large\bfseries\sffamily,
		inner sep =4pt, shape =rectangle, line width =1.2pt,text width =.9\textwidth, align = center
		]
		{#4}
		;
	}
}



%%%=============== Lệnh tạo mũi tên =====================%%%
%%%\muiten[up][down][length][độ cao][style] %[up][down][length][độ cao] có thể bỏ trống
%\NewDocumentCommand{\muiten}{O{}O{}O{}O{}O{-latex}m}{
%	\ifblank{#1}{\def\tuychonone{}}{\def\tuychonone{#1}}
%	\ifblank{#2}{\def\tuychontwo{}}{\def\tuychontwo{#2}}
%	\ifblank{#3}{\def\tuychonthree{.7}}{\def\tuychonthree{#3}}
%	\ifblank{#4}{\def\tuychonfour{0pt}}{\def\tuychonfour{#4}}
%	\setchemfig{arrow offset=2pt,arrow double sep =2.45pt}
%	\protect\mbox{\schemestart\vphantom{1}%
%		\arrow{#6[\scriptsize\tuychonone][\scriptsize\tuychontwo][\tuychonfour]}[0,\tuychonthree,#5]
%		\vphantom{1}\schemestop}
%}
%
%%%==================Lệnh tạo phương trình phản ứng =====================%%%
%%============\puhh[up][down][length][độ cao][style]{A \+ B\+ C}{-> %<=>}{D \+ E}====%%%
%\NewDocumentCommand{\puhh}{O{}O{}O{}O{}O{-latex}mmm}{
%	\ifblank{#1}{\def\tuychonone{}}{\def\tuychonone{#1}}
%	\ifblank{#2}{\def\tuychontwo{}}{\def\tuychontwo{#2}}
%	\ifblank{#3}{\def\tuychonthree{.7}}{\def\tuychonthree{#3}}
%	\ifblank{#4}{\def\tuychonfour{0pt}}{\def\tuychonfour{#4}}
%	\setchemfig{arrow double sep =2.45pt,arrow offset=6pt}
%	\schemestart
%	#6\arrow{#7[\scriptsize\tuychonone][\scriptsize\tuychontwo][\tuychonfour]}[0,\tuychonthree,#5]#8
%	\schemestop}

\NewDocumentCommand{\AOs}{O{1.8}O{}O{}O{}}{%
	\ifblank{#2}{\def\tuychontwo{\maunhan}}{\def\tuychontwo{#2}}
	\ifblank{#3}{\def\tuychonthree{.80}}{\def\tuychonthree{#3}}
	\ifblank{#4}{\def\tuychonfour{densely dotted}}{\def\tuychonfour{#4}}
	\begin{tikzpicture}[declare function={d=#1cm;r=.65*d;h=0.125*d;R=.36*d;},node font=\bfseries\sffamily]
		\tikzstyle{linestyle} = [line width=0.015*d,\tuychontwo!80,\tuychonfour]
		\tikzstyle{myshapestyle}=[ball color = \tuychontwo,opacity=\tuychonthree]
		\draw[linestyle] ([xshift=-1.2*R]0*d,0)--([xshift=1.2*R]0*d,0);
		\fill[myshapestyle] (0*d,0) circle (R);
	\end{tikzpicture}
}

\NewDocumentCommand{\AOp}{O{1.8}O{}O{}O{}}{%
	\ifblank{#2}{\def\tuychontwo{\maunhan}}{\def\tuychontwo{#2}}
	\ifblank{#3}{\def\tuychonthree{.80}}{\def\tuychonthree{#3}}
	\ifblank{#4}{\def\tuychonfour{densely dotted}}{\def\tuychonfour{#4}}
	\begin{tikzpicture}[declare function={d=2cm;r=.65*d;h=0.125*d;R=.36*d;},node font=\bfseries\sffamily]
		\tikzstyle{linestyle} = [line width=0.015*d,\maunhan!80,\tuychonfour,>=stealth]
		\tikzstyle{myshapestyle} = [ball color = \tuychontwo,opacity=\tuychonthree]
		\draw[linestyle] ([xshift=-1.1*d]2*d,0)--([xshift=1.1*d]2*d,0);
		\path[myshapestyle] (1*d,0)..controls +(90:r) and +(90:{.25*r})..(2*d,0)--
		(2*d,0)..controls +(-90:{.25*r}) and +(-90:r)..(1*d,0);
		\path[myshapestyle] (3*d,0)..controls +(90:r) and +(90:{.25*r})..(2*d,0)--
		(2*d,0)..controls +(-90:{.25*r}) and +(-90:r)..(3*d,0);
	\end{tikzpicture}
}



\tikzset{
	pics/.cd,
	AOPa/.style={code={
		\draw[linestyle] (0,{-d - h})--(0,{d + h});
		\path[myshapestyle] (0,0)..controls +(0:{.25*r}) and +(0:r)..(0,d)--
		(0,d)..controls +(180:r) and +(180:{.25*r})..(0,0);
		\path[myshapestyle] (0,-d)..controls +(180:r) and +(180:{.25*r})..(0,0)--
		(0,0)..controls +(0:{.25*r}) and +(0:r)..(0,-d);
						}},
	AOPb/.style={code={
		\draw[linestyle] (0,{-d - h})--(0,{d + h});
		\path[myshapestyle,ball color = \maunhan!90] (0,0)..controls +(0:{.25*r}) and +(0:r)..(0,d)--
		(0,d)..controls +(180:r) and +(180:{.25*r})..(0,0);
		\path[myshapestyle,ball color = \maunhan!90] (0,-d)..controls +(180:r) and +(180:{.25*r})..(0,0)--
		(0,0)..controls +(0:{.25*r}) and +(0:r)..(0,-d);
			}},
	muiten/.style={code={
		\path[draw,->,-latex,line width=.035*d,\maunhan!80] (0,0)--(3,0);
	}}
}
