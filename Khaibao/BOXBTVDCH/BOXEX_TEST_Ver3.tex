%%%===========HỘP BÀI TẬP TRẮC NGHIỆM TỰ LUẬN=================%%%
\newcommand\circl[2][\mycolor]{\tikz[baseline=(char.base)]{\node[shape=circle,inner sep=1pt,draw=#1,fill=#1!15,font=\color{\mycolor!50!black}\bfseries\fontfamily{qag}\selectfont,minimum size=15pt] (char) {#2};
	}
}
\usepackage{ex_test}
\usepackage{ifthen}
\renewcommand{\FalseEX}{\stepcounter{dapan}{{\circl{\Alph{dapan}\dotEX}}}}
\renewcommand{\TrueEX}{\stepcounter{dapan}{{\circl{\Alph{dapan}\dotEX}}}}
%\renewcommand{\immini}[3][]{%
%	\tcbsidebyside[
%	sidebyside adapt=right,
%	blanker,sidebyside gap=5mm,
%	sidebyside align=top seam
%	]{%
%		#2
%	}{%
%		#3
%	}
%}
%\renewcommand{\hinhphai}[2]{%
%	\tcbsidebyside[
%	sidebyside adapt=right,
%	blanker,sidebyside gap=3mm,
%	sidebyside align=top seam,
%	]{%
%		#1
%	}{%
%		#2
%	}
%}
\def\tieudehinh{}
\newcommand{\hinhphai}[2]{%
	\tcbsidebyside[
	sidebyside adapt=right,
	blanker,sidebyside gap=5mm,
	sidebyside align=top seam,
	]{%
		{\tieudehinh}%
		#1
	}{%
		#2%
	}
}
%%%================Lệnh immini================%%%
\def\loaicau{}
\renewcommand{\immini}[3][]{
	\savebox{\imbox}{#3}
	\setlength{\widthPB}{0.55\linewidth}
	
	\tcbsidebyside[
	sidebyside adapt=right,
	blanker,sidebyside gap=5mm,
	sidebyside align=top seam,
	]{%
		{\tieudehinh}#2
	}{%
		\usebox{\imbox}
	}
	\loaicau
}
%%%%%%%%%%%%%%%%%%%%%%Tách lời giải%%%%%%%%%%%%%%%%%%%%%%%%%%%%%%%%%%%%%%%%
\Newassociation{giaibth}{loigiaibth}{ansbth}
\Newassociation{giaibt}{loigiaibt}{ansbt}
\Newassociation{giaiex}{loigiaiex}{ansex}
\NewTColorBox{ansbt}{m}{
	breakable,
	enhanced,
	outer arc=0pt,
	arc=0pt,
	colframe=white,
	frame hidden,
	left=-6pt,right=0pt,top=0pt,
	colback=white,
	attach boxed title to top left,
	boxed title style={
		colback=white,
		outer arc=0pt,
		arc=0pt,
		top=1pt,
		bottom=1pt,
		left=3pt,
		right=3pt,
		colframe=white
	},
	fonttitle=\bfseries\sffamily\selectfont\color{white},
	title={HDBT~#1},
	overlay unbroken and first={
		\path (title.north west) coordinate (A)
		($ (title.south west) +(-4pt,0pt)$) coordinate (B)
		(title.south east) coordinate (C)
		($ (title.north east)+(4pt,0) $) coordinate (D);
		\path[rounded corners=2pt,fill=\mycolor,
		preaction={transform canvas={shift={(-45:2pt)}},left color=\mycolor!45,right color=\mycolor!25}] 
		(A)--(B)--(C)--(D)--cycle;
		\path (A)--(C) node[midway,font=\color{white}\bfseries\sffamily\selectfont](Bai){\textsl{HDBT~#1}};
		\draw[rounded corners=2pt,thick,\mycolor] ([xshift=-3pt]B) coordinate (Bt)
		--([shift={(-2pt,2pt)}]A)--+(\linewidth,0) coordinate (Ct);
		\fill[\mycolor] (Bt) circle (1pt) (Ct) circle (2pt);
	}
}

\NewTColorBox{ansex}{m}{
	breakable,
	enhanced,
	outer arc=0pt,
	arc=0pt,
	colframe=white,
	frame hidden,
	left=-6pt,right=0pt,top=0pt,
	colback=white,
	attach boxed title to top left,
	boxed title style={
		colback=white,
		outer arc=0pt,
		arc=0pt,
		top=1pt,
		bottom=1pt,
		left=3pt,
		right=3pt,
		colframe=white
	},
	fonttitle=\bfseries\sffamily\selectfont\color{white},
	title={HD Câu~#1},
	overlay unbroken and first={
		\path (title.north west) coordinate (A)
		($ (title.south west) +(-4pt,0pt)$) coordinate (B)
		(title.south east) coordinate (C)
		($ (title.north east)+(4pt,0) $) coordinate (D);
		\path[rounded corners=2pt,fill=\mycolor,
		preaction={transform canvas={shift={(-45:2pt)}},left color=\mycolor!45,right color=\mycolor!25}] 
		(A)--(B)--(C)--(D)--cycle;
		\path (A)--(C) node[midway,font=\color{white}\bfseries\sffamily\selectfont](Bai){\textsl{HD Câu~#1}};
		\draw[rounded corners=2pt,thick,\mycolor] ([xshift=-3pt]B) coordinate (Bt)
		--([shift={(-2pt,2pt)}]A)--+(\linewidth,0) coordinate (Ct);
		\fill[\mycolor] (Bt) circle (1pt) (Ct) circle (2pt);
	}
}
\def\ansexhead{\begin{ansex}}
\def\ansexend{\end{ansex}}
\renewenvironment{loigiaibth}[1]{\begin{ansbt}{#1}}{\end{ansbt}}
%\renewenvironment{loigiaibt}[1]{\begin{ansbt}{#1}}{\end{ansbt}}
\renewenvironment{loigiaiex}[1]{\ansexhead{#1}}{\ansexend}
%\renewenvironment{loigiaibtn}[1]{\begin{ansbt}{#1}}{\end{ansbt}}
\def\luuloigiaibt{
	\AtBeginEnvironment{bt}{
		\renewcommand{\loigiai}[1]{%
		\end{btbox}
		\def\btend{}
		\scantokens{%
			\begin{giaibth}
				##1
			\end{giaibth}}
	}
}
}

\def\luuloigiaiex{
\AtBeginEnvironment{ex}{
	\renewcommand{\loigiai}[1]{%
		\ifnum\the\value{numTrue}=1
		\scantokens{%
			\begin{giaiex}
				##1
				\phantom{a}\hfill{\bfseries\sffamily Chọn~\circleTrue{A}}
		\end{giaiex}}
		\fi%
		\ifnum\the\value{numTrue}=2
		\scantokens{%
			\begin{giaiex}
				##1
				\phantom{a}\hfill{\bfseries\sffamily Chọn~\circleTrue{B}}
		\end{giaiex}}
		\fi%	    
		\ifnum\the\value{numTrue}=3
		\scantokens{%
			\begin{giaiex}
				##1
				\phantom{a}\hfill{\bfseries\sffamily Chọn~\circleTrue{C}}
		\end{giaiex}}
		\fi  %	   
		\ifnum\the\value{numTrue}=4
		\scantokens{%
			\begin{giaiex}
				##1
				\phantom{a}\hfill{\bfseries\sffamily Chọn~\circleTrue{D}}
		\end{giaiex}}
		\fi  
	}  
}
}


\def\luuloigiaiex{
	\AtBeginEnvironment{ex}{
		\setboolean{TNdungsai}{false}%
		\gdef\loaicau{\setboolean{TNdungsai}{false}}%
		\renewcommand{\loigiai}[1]{%
			\ifthenelse{\boolean{TNdungsai}}{%Câu hỏi trắc nghiệm đúng sai
				\scantokens{%
					\begin{giaiex}
						##1
						\phantom{a}\hfill{{\faKey}~ 
							\ifbool{Ads}{\TLdung{A}}{\TLsai{A}}~
							\ifbool{Bds}{\TLdung{B}}{\TLsai{B}}~
							\ifbool{Cds}{\TLdung{C}}{\TLsai{C}}~
							\ifbool{Dds}{\TLdung{D}}{\TLsai{D}} 
						}%
				\end{giaiex}}
				\Writetofile{ansbook}{\string\def\string\writeANS{\saveans}}
				\scantokens{
					\begin{solbook}
						\writeANS
					\end{solbook}
				}%
			}{%Câu hỏi trắc nghiệm 1 phương án
				\ifnum\the\value{numTrue}=1
				\scantokens{%
					\begin{giaiex}
						##1
						\phantom{a}\hfill{ \faKey~\circlenum{A}}
				\end{giaiex}}
				\fi%
				\ifnum\the\value{numTrue}=2
				\scantokens{%
					\begin{giaiex}
						##1
						\phantom{a}\hfill{ \faKey~\circlenum{B}}
				\end{giaiex}}
				\fi%	 
				\ifnum\the\value{numTrue}=3
				\scantokens{%
					\begin{giaiex}
						##1
						\phantom{a}\hfill{ \faKey~\circlenum{C}}
				\end{giaiex}}
				\fi %	 
				\ifnum\the\value{numTrue}=4
				\scantokens{%
					\begin{giaiex}
						##1
						\phantom{a}\hfill{ \faKey~\circlenum{D}}
				\end{giaiex}}
				\fi\vspace*{-3pt}%
			} 
		}
	}
}


\def\luulgEXTF{
	\AtBeginEnvironment{ex}{
		\renewcommand{\loigiai}[1]{%
			\Writetofile{ansex}{\string\def\string\writeANS{\saveans}}
			\scantokens{%
				\begin{giaiex}
					##1
					\phantom{a}\hfill{ \faKey~\writeANS}
				\end{giaiex}
			}% 
			\Writetofile{ansbook}{\string\def\string\writeANS{\saveans}}
			\scantokens{
				\begin{solbook}
					\writeANS
				\end{solbook}%
			} 
		}
	}
}







%%%%%%%%%%%%%%%%%%%%%%%%%%%%%%%%%%%%%%%%%%%%%%%%%%%%%%%%%%%%%%%%%%%%%%%%%%%%%%%
%%%%%%%%%%%%%%%%%%%%%%%%%%%%%%%%HỘP CÂU CÓ 3 TÙY CHỌN%%%%%%%%%%%%%%%%%%%%%%%%%%%%%
\newbool{ktex}
\newcommand{\sodongkeex}[1][]{\global\setbool{ktex}{true}\end{exbox}\def\exend{} \noindent\textcolor{\mycolor}{\reflectbox{\Large\WritingHand}\ {\fmmfamily\LARGE Bài làm:}}\taodongke{#1}}
%\newcounter{ex}
\NewTColorBox[auto counter]{exbox}{+!O{}O{}O{}}{%
enhanced,
breakable,
toprule at break=-\tcboxedtitleheight,
fonttitle=\bfseries\sffamily\color{white},
title={\faClockO\ Câu~\theex},
%colframe=\mycolor!65!black,
empty,opacityback=0,
colback=white,
colbacktitle=white,
fonttitle=\bfseries,
coltitle=black,
attach boxed title to top left=
{yshift=-2mm-\tcboxedtitleheight,yshifttext=2mm-\tcboxedtitleheight/2},
boxed title style={
frame hidden,
outer arc=0pt,
arc=0pt,
bottom=3pt,
left=0pt,
right=0pt
},
overlay unbroken ={
\path[draw=\mycolor,rounded corners=5pt,thick] (frame.north west) rectangle (frame.south east);
\path
($ (title.north west) +(-2pt,0pt)$) coordinate (A)
($ (title.south west) +(-2pt,3pt)$) coordinate (B)
($ (title.south east)+(3pt,3pt) $) coordinate (C)
($ (title.north east)+(3pt,0) $) coordinate (D)
($ (A)+(0.99*\linewidth,0) $) coordinate (E)
;
\path[fill=\mycolor!15,rounded corners=3pt]
(A)--(B)--([xshift=0.85*\linewidth]C) coordinate (E)--([xshift=0.85*\linewidth]D) coordinate (F)--cycle;
\path[fill=\mycolor!70!black,rounded corners=2pt]
(A)--(B)--([xshift=3pt]C)--([xshift=3pt]D)--cycle;
\path[left color=\mycolor,right color=\mycolor,rounded corners=3pt]
([xshift=-2pt]A)--([xshift=-2pt]B)--(C)--(D)--cycle;
\path ($ (A)!0.5!(B) +(0pt,0)$) node[anchor=west,font=\color{white}\bfseries\sffamily\selectfont]{{\faClockO\ Câu~\theex}};
\path ($ (C)!0.5!(D) +(9pt,0)$) node[anchor=west,font=\color{\mycolor}]{\taosao{#2}};
\path ($ (E)!0.5!(F) +(-9pt,0)$) node[anchor=east,font=\color{\mycolor!70!black}\itshape\bfseries]{#1};
\path ([shift={(-1pt,7pt)}]frame.south east) node[anchor=east,font=\fontsize{9.2 pt}{1pt}\selectfont\color{\mycolor!70!black}\itshape\bfseries]{#3};
},
overlay first={
\path[draw=\mycolor,rounded corners=5pt, thick] (frame.south west) coordinate (FSW)
--(frame.north west) coordinate (FNW)
-- (frame.north east) coordinate (FNE)
--(frame.south east) coordinate (FSE)
(FSW)--(FSE)
;
\path
($ (title.north west) +(-2pt,0pt)$) coordinate (A)
($ (title.south west) +(-2pt,3pt)$) coordinate (B)
($ (title.south east)+(3pt,3pt) $) coordinate (C)
($ (title.north east)+(3pt,0) $) coordinate (D)
($ (A)+(0.99*\linewidth,0) $) coordinate (E)
;
\path[fill=\mycolor!15,rounded corners=3pt]
(A)--(B)--([xshift=0.85*\linewidth]C) coordinate (E)--([xshift=0.85*\linewidth]D) coordinate (F)--cycle;
\path[fill=\mycolor!70!black,rounded corners=2pt]
(A)--(B)--([xshift=3pt]C)--([xshift=3pt]D)--cycle;
\path[left color=\mycolor,right color=\mycolor,rounded corners=3pt]
([xshift=-2pt]A)--([xshift=-2pt]B)--(C)--(D)--cycle;
\path ($ (A)!0.5!(B) +(0pt,0)$) node[anchor=west,font=\color{white}\bfseries\sffamily\selectfont]{{\faClockO\ Câu~\theex}};
\path ($ (C)!0.5!(D) +(9pt,0)$) node[anchor=west,font=\color{\mycolor}]{\taosao{#2}};
\path ($ (E)!0.5!(F) +(-9pt,0)$) node[anchor=east,font=\color{\mycolor!70!black}\itshape\bfseries]{#1};
},
overlay middle ={
\path[draw=\mycolor,thick] (frame.north west) rectangle (frame.south east);
},
overlay  last ={
\path[draw=\mycolor,rounded corners=5pt, thick] (frame.north west) coordinate (FNW)-- (frame.south west) coordinate (FSW)
--(frame.south east) coordinate (FSE)--(frame.north east) coordinate (FNE)
(FNW)--(FNE)
;
\path ([shift={(-1pt,7pt)}]frame.south east) node[anchor=east,font=\fontsize{9.2 pt}{1pt}\selectfont\color{\mycolor!70!black}\itshape\bfseries]{#3};
},
top=\tcboxedtitleheight
}
\def\exhead#1#2#3{\begin{exbox}[#1][#2][#3]}
\def\exend{\end{exbox}}
\RenewDocumentEnvironment{ex}{+!O{}O{}O{}}{
\ifblank{#1}{\def\tuychonone{}}{\def\tuychonone{#1}}
\ifblank{#2}{\def\tuychontwo{0}}{\def\tuychontwo{#2}}
\ifblank{#3}{\def\tuychonthree{}}{\def\tuychonthree{#3}}
\exhead{\tuychonone}{\tuychontwo}{\tuychonthree}
}{\exend}
\AtBeginEnvironment{ex}{%
\renewcommand{\FalseEX}{\stepcounter{dapan}{\circl{\Alph{dapan}}}}
\renewcommand{\TrueEX}{\stepcounter{dapan}{\circl{\Alph{dapan}}}}
\refstepcounter{ex} % tăng số điếm bt lên 1
\global\setbool{ktex}{false}
\setcounter{numTrue}{0}%
\setcounter{numTrueSol}{0}%
\renewcommand{\loigiai}[1]{%
\ifbool{ktex}{}{
\end{exbox}
\def\exend{}
%%==================Tính số dòng của lời giải==================%%%
\setbox0=\vbox{\hbox{
		\noindent\begin{minipage}{\linewidth}%
			#1aaaaaaaaaaa
		\end{minipage}
}}
\linepar=\ht0
%	\pgfmathparse{int((\linepar-\fboxsep)/(\lineheight))}
\pgfmathparse{int((\linepar-\fboxsep)/(\lineheight)+3)}
\noindent\textcolor{\mycolor!70!black}{\reflectbox{\Large\WritingHand}\ {\fmmfamily\LARGE Bài làm:}}\taodongke{\pgfmathresult}
}
}
\renewcommand{\huongdan}[1]{%
\par\noindent	
\end{exbox}
\def\exend{}
\par\noindent%
{\color{\mycolor!50!black}\reflectbox{\Large\WritingHand}\ {\fmmfamily\LARGE \textbf{Hướng dẫn giải:}}} 
\par\noindent
#1
\ifthenelse{\value{numTrueSol}>0}{
\phantom{a}\hfill\textcolor{\mycolor!50!black}{\bfseries\sffamily Chọn~\circleTrue{\Alph{numTrueSol}}}
}{}
}
}

%%%%%%%%%%%%%%%%%%%%%%%%%%%%%%%HỘP Bai Tap CÓ 3 TÙY CHỌN%%%%%%%%%%%%%%%%%%%%%%%%%%%%%
\newcommand{\huongdan}[1]{}%tạo tạm lệnh hướng dẫn
\newbool{ktbt}
\newcommand{\sodongkebt}[1][]{\global\setbool{ktbt}{true}\end{btbox}\def\btend{} \noindent\textcolor{\mycolor}{\reflectbox{\Large\WritingHand}\ {\fmmfamily\LARGE Bài làm:}}\taodongke{#1}}
\newcounter{bt}
\NewTColorBox{btbox}{+!O{}O{}O{}}{%
enhanced,
breakable,
toprule at break=-\tcboxedtitleheight,
fonttitle=\bfseries\sffamily\color{white},
title={\faClockO\ Bài~\thebt},
%colframe=\mycolor!65!black,
empty,opacityback=0,
colback=white,
colbacktitle=white,
fonttitle=\bfseries,
coltitle=black,
attach boxed title to top left=
{yshift=-2mm-\tcboxedtitleheight,yshifttext=2mm-\tcboxedtitleheight/2},
boxed title style={
frame hidden,
outer arc=0pt,
arc=0pt,
top=3pt,
bottom=3pt,
left=0pt,
right=0pt
},
overlay unbroken ={
\path[draw=\mycolor,rounded corners=5pt, thick] (frame.north west) rectangle (frame.south east);
\path
($ (title.north west) +(-2pt,0pt)$) coordinate (A)
($ (title.south west) +(-2pt,3pt)$) coordinate (B)
($ (title.south east)+(3pt,3pt) $) coordinate (C)
($ (title.north east)+(3pt,0) $) coordinate (D)
($ (A)+(0.99*\linewidth,0) $) coordinate (E)
;
\path[fill=\mycolor!15,rounded corners=3pt]
(A)--(B)--([xshift=0.85*\linewidth]C) coordinate (E)--([xshift=0.85*\linewidth]D) coordinate (F)--cycle;
\path[fill=\mycolor!70!black,rounded corners=2pt]
(A)--(B)--([xshift=3pt]C)--([xshift=3pt]D)--cycle;
\path[left color=\mycolor,right color=\mycolor,rounded corners=3pt]
([xshift=-2pt]A)--([xshift=-2pt]B)--(C)--(D)--cycle;
\path ($ (A)!0.5!(B) +(0pt,0)$) node[anchor=west,font=\color{white}\bfseries\sffamily\selectfont]{{\faClockO\ Bài~\thebt}};
\path ($ (C)!0.5!(D) +(9pt,0)$) node[anchor=west,font=\color{\mycolor}]{\taosao{#2}};
\path ($ (E)!0.5!(F) +(-9pt,0)$) node[anchor=east,font=\color{\mycolor!70!black}\itshape\bfseries]{#1};
\path ([shift={(-1pt,7pt)}]frame.south east) node[anchor=east,font=\fontsize{9.2 pt}{1pt}\selectfont\color{\mycolor!70!black}\itshape\bfseries]{#3};
},
overlay first  ={
\path[draw=\mycolor,rounded corners=5pt, thick] (frame.south west) coordinate (FSW)
--(frame.north west) coordinate (FNW)
-- (frame.north east) coordinate (FNE)
--(frame.south east) coordinate (FSE)
(FSW)--(FSE)
;
\path
($ (title.north west) +(-2pt,0pt)$) coordinate (A)
($ (title.south west) +(-2pt,3pt)$) coordinate (B)
($ (title.south east)+(3pt,3pt) $) coordinate (C)
($ (title.north east)+(3pt,0) $) coordinate (D)
($ (A)+(0.99*\linewidth,0) $) coordinate (E)
;
\path[fill=\mycolor!15,rounded corners=3pt]
(A)--(B)--([xshift=0.85*\linewidth]C) coordinate (E)--([xshift=0.85*\linewidth]D) coordinate (F)--cycle;
\path[fill=\mycolor!70!black,rounded corners=2pt]
(A)--(B)--([xshift=3pt]C)--([xshift=3pt]D)--cycle;
\path[left color=\mycolor,right color=\mycolor,rounded corners=3pt]
([xshift=-2pt]A)--([xshift=-2pt]B)--(C)--(D)--cycle;
\path ($ (A)!0.5!(B) +(0pt,0)$) node[anchor=west,font=\color{white}\bfseries\sffamily\selectfont]{{\faClockO\ Bài~\thebt}};
\path ($ (C)!0.5!(D) +(9pt,0)$) node[anchor=west,font=\color{\mycolor}]{\taosao{#2}};
\path ($ (E)!0.5!(F) +(-9pt,0)$) node[anchor=east,font=\color{\mycolor!70!black}\itshape\bfseries]{#1};
},
overlay middle ={
\path[draw=\mycolor,thick] (frame.north west) rectangle (frame.south east);
},
overlay last ={
\path[draw=\mycolor,rounded corners=5pt, thick] (frame.north west) coordinate (FNW)-- (frame.south west) coordinate (FSW)
--(frame.south east) coordinate (FSE)--(frame.north east) coordinate (FNE)
(FNW)--(FNE)
;
\path ([shift={(-1pt,7pt)}]frame.south east) node[anchor=east,font=\fontsize{9.2 pt}{1pt}\selectfont\color{\mycolor!70!black}\itshape\bfseries]{#3};
},
top=\tcboxedtitleheight
}
\def\bthead#1#2#3{\begin{btbox}[#1][#2][#3]}
\def\btend{\end{btbox}}
\NewDocumentEnvironment{bt}{+!O{}O{}O{}}{
\ifblank{#1}{\def\tuychonone{}}{\def\tuychonone{#1}}
\ifblank{#2}{\def\tuychontwo{0}}{\def\tuychontwo{#2}}
\ifblank{#3}{\def\tuychonthree{}}{\def\tuychonthree{#3}}
\bthead{\tuychonone}{\tuychontwo}{\tuychonthree}
}{\btend}
\AtBeginEnvironment{bt}{%
\renewcommand{\FalseEX}{\stepcounter{dapan}{\circl{\Alph{dapan}}}}
\renewcommand{\TrueEX}{\stepcounter{dapan}{\circl{\Alph{dapan}}}}
\refstepcounter{bt} % tăng số điếm bt lên 1
\global\setbool{ktbt}{false}
\setcounter{numTrue}{0}%
\setcounter{numTrueSol}{0}%
\renewcommand{\loigiai}[1]{%
\ifbool{ktbt}{}{
\end{btbox}
\def\btend{}
%%==================Tính số dòng của lời giải==================%%%
\setbox0=\vbox{\hbox{
\noindent\begin{minipage}{\linewidth}%
#1aaaaaaaaaaa
\end{minipage}
}}
\linepar=\ht0
%	\pgfmathparse{int((\linepar-\fboxsep)/(\lineheight))}
\pgfmathparse{int((\linepar-\fboxsep)/(\lineheight)+3)}
\noindent\textcolor{\mycolor!70!black}{\reflectbox{\Large\WritingHand}\ {\fmmfamily\LARGE Bài làm:}}\taodongke{\pgfmathresult}
}
}
\renewcommand{\huongdan}[1]{%
\par\noindent	
\end{btbox}
\def\btend{}
\par\noindent%
{\color{\mycolor!50!black}\reflectbox{\Large\WritingHand}\ {\fmmfamily\LARGE \textbf{Hướng dẫn giải:}}} 
\par\noindent
#1
\ifthenelse{\value{numTrueSol}>0}{
\phantom{a}\hfill\textcolor{\mycolor!50!black}{\bfseries\sffamily Chọn~\circleTrue{\Alph{numTrueSol}}}
}{}
}
}

%%====================Hộp Ví dụ có 3 tùy chọn==========================%%%
\newbool{ktvd}
\newcommand{\sodongkevd}[1][]{\global\setbool{ktvd}{true}\end{vdbox}\def\vdend{} \noindent\textcolor{\mycolor}{\reflectbox{\Large\WritingHand}\ {\fmmfamily\LARGE Bài làm:}}\taodongke{#1}}
\newcounter{vd}
\NewTColorBox{vdbox}{+!O{}O{}O{}}{%
enhanced,
breakable,
left=6pt,
toprule at break=-\tcboxedtitleheight,
fonttitle=\bfseries\sffamily\color{white},
title={\faClockO\ Ví dụ~\thevd},
%colframe=\mycolor!65!black,
colback=white,
empty,
opacityback=0,
colbacktitle=white,
fonttitle=\bfseries,
coltitle=black,
code={\refstepcounter{vd}},
attach boxed title to top left=
{yshift=-2mm-\tcboxedtitleheight,yshifttext=2mm-\tcboxedtitleheight/2},
boxed title style={
frame hidden,
outer arc=0pt,
arc=0pt,
top=2pt,
bottom=2pt,
left=0pt,
right=0pt
},
overlay unbroken ={
\path[draw=\mycolor,rounded corners=5pt,ultra thick] (frame.north west) rectangle (frame.south east);
\path
($ (title.north west) +(-2pt,0pt)$) coordinate (A)
($ (title.south west) +(-2pt,3pt)$) coordinate (B)
($ (title.south east)+(3pt,3pt) $) coordinate (C)
($ (title.north east)+(3pt,0) $) coordinate (D)
($ (A)+(0.99*\linewidth,0) $) coordinate (E)
;
\path[fill=\mycolor!15,rounded corners=3pt]
(A)--(B)--([xshift=0.86*\linewidth]C) coordinate (E)--([xshift=0.86*\linewidth]D) coordinate (F)--cycle;
\path[fill=\mycolor!70!black,rounded corners=2pt]
(A)--(B)--([xshift=3pt]C)--([xshift=3pt]D)--cycle;
\path[left color=\mycolor,right color=\mycolor,rounded corners=3pt]
([xshift=-2pt]A)--([xshift=-2pt]B)--(C)--(D)--cycle;
\path ($ (A)!0.5!(B) +(0pt,0)$) node[anchor=west,font=\color{white}\bfseries\sffamily\selectfont]{{\faClockO\ Ví dụ~\thevd}};
\path ($ (C)!0.5!(D) +(9pt,0)$) node[anchor=west,font=\color{\mycolor}]{\taosao{#2}};
\path ($ (E)!0.5!(F) +(-9pt,0)$) node[anchor=east,font=\color{\mycolor!70!black}\itshape\bfseries]{#1};
\path ([shift={(-1pt,7pt)}]frame.south east) node[anchor=east,font=\fontsize{9.2 pt}{1pt}\selectfont\color{\mycolor!70!black}\itshape\bfseries]{#3};
},
overlay  first={
\path[draw=\mycolor,rounded corners=5pt, thick] (frame.south west) coordinate (FSW)
--(frame.north west) coordinate (FNW)
-- (frame.north east) coordinate (FNE)
--(frame.south east) coordinate (FSE)
(FSW)--(FSE)
;
\path
($ (title.north west) +(-2pt,0pt)$) coordinate (A)
($ (title.south west) +(-2pt,3pt)$) coordinate (B)
($ (title.south east)+(3pt,3pt) $) coordinate (C)
($ (title.north east)+(3pt,0) $) coordinate (D)
($ (A)+(0.99*\linewidth,0) $) coordinate (E)
;
\path[fill=\mycolor!15,rounded corners=3pt]
(A)--(B)--([xshift=0.86*\linewidth]C) coordinate (E)--([xshift=0.86*\linewidth]D) coordinate (F)--cycle;
\path[fill=\mycolor!70!black,rounded corners=2pt]
(A)--(B)--([xshift=3pt]C)--([xshift=3pt]D)--cycle;
\path[left color=\mycolor,right color=\mycolor,rounded corners=3pt]
([xshift=-2pt]A)--([xshift=-2pt]B)--(C)--(D)--cycle;
\path ($ (A)!0.5!(B) +(0pt,0)$) node[anchor=west,font=\color{white}\bfseries\sffamily\selectfont]{{\faClockO\ Ví dụ~\thevd}};
\path ($ (C)!0.5!(D) +(9pt,0)$) node[anchor=west,font=\color{\mycolor}]{\taosao{#2}};
\path ($ (E)!0.5!(F) +(-9pt,0)$) node[anchor=east,font=\color{\mycolor!70!black}\itshape\bfseries]{#1};
},
overlay middle ={
\path[draw=\mycolor, thick] (frame.north west) rectangle (frame.south east);
},
overlay last ={
\path[draw=\mycolor,rounded corners=5pt, thick] (frame.north west) coordinate (FNW)-- (frame.south west) coordinate (FSW)
--(frame.south east) coordinate (FSE)--(frame.north east) coordinate (FNE)
(FNW)--(FNE)
;
\path ([shift={(-1pt,7pt)}]frame.south east) node[anchor=east,font=\fontsize{9.2 pt}{1pt}\selectfont\color{\mycolor!70!black}\itshape\bfseries]{#3};
},
top=\tcboxedtitleheight
}
\def\vdhead#1#2#3{\begin{vdbox}[#1][#2][#3]}
\def\vdend{\end{vdbox}}
\NewDocumentEnvironment{vd}{+!O{}O{}O{}}{%
\ifblank{#1}{\def\tuychonone{}}{\def\tuychonone{#1}}
\ifblank{#2}{\def\tuychontwo{0}}{\def\tuychontwo{#2}}
\ifblank{#3}{\def\tuychonthree{}}{\def\tuychonthree{#3}}
\vdhead{\tuychonone}{\tuychontwo}{\tuychonthree}
}{\vdend}
\AtBeginEnvironment{vd}{%
\renewcommand{\FalseEX}{\stepcounter{dapan}{\circl{\Alph{dapan}}}}
\renewcommand{\TrueEX}{\stepcounter{dapan}{\circl{\Alph{dapan}}}}
%\refstepcounter{vd} % tăng số điếm bt lên 1
\global\setbool{ktvd}{false}
\setcounter{numTrue}{0}%
\setcounter{numTrueSol}{0}%
\renewcommand{\loigiai}[1]{%
\ifbool{ktvd}{}{
\end{vdbox}
\def\vdend{}
%%==================Tính số dòng của lời giải==================%%%
\setbox0=\vbox{\hbox{
\noindent\begin{minipage}{\linewidth}%
#1
\end{minipage}
}}
\linepar=\ht0
%		\pgfmathparse{int((\linepar-\fboxsep)/(\lineheight))}
\pgfmathparse{int((\linepar-\fboxsep)/(\lineheight)+3)}
\noindent\textcolor{\mycolor!70!black}{\reflectbox{\Large\WritingHand}\ {\fmmfamily\LARGE Bài làm:}}\taodongke{\pgfmathresult}
}
}
\renewcommand{\huongdan}[1]{%
\end{vdbox}
\def\vdend{}
{\par\noindent\color{\mycolor!50!black}\reflectbox{\Large\WritingHand}\ {\fmmfamily\LARGE \textbf{Hướng dẫn giải:}}}
#1
\ifthenelse{\value{numTrueSol}>0}{
\phantom{a}\hfill\textcolor{\mycolor!50!black}{\bfseries\sffamily Chọn~\circleTrue{\Alph{numTrueSol}}}
}{}
}
}
\newcommand\circlenum[2][\mycolor]{\tikz[baseline=(char.base)]
{\node[shape=circle,inner sep=1pt,draw=#1,fill=#1!10,
font=\footnotesize\bfseries\fontfamily{qag}\selectfont,minimum size=14pt,outer sep=0pt] (char) {#2};}}


%%%==================Hộp vdex 3 tùy chọn==================%%%
\newcounter{vdex}
\NewTColorBox{vdexbox}{+!O{}O{}O{}}{%
enhanced,
breakable,
left=6pt,
toprule at break=-\tcboxedtitleheight,
fonttitle=\bfseries\sffamily\color{white},
title={\faClockO\ Ví dụ~\thevd},
%colframe=\mycolor!65!black,
empty,opacityback=0,
colback=white,
colbacktitle=white,
fonttitle=\bfseries,
coltitle=black,
code={\refstepcounter{vd}},
attach boxed title to top left=
{yshift=-2mm-\tcboxedtitleheight,yshifttext=2mm-\tcboxedtitleheight/2},
boxed title style={
frame hidden,
outer arc=0pt,
arc=0pt,
top=3pt,
bottom=3pt,
left=0pt,
right=0pt
},
overlay unbroken ={
\path[draw=\mycolor,rounded corners=5pt, thick] (frame.north west) rectangle (frame.south east);
\path
($ (title.north west) +(-2pt,0pt)$) coordinate (A)
($ (title.south west) +(-2pt,3pt)$) coordinate (B)
($ (title.south east)+(3pt,3pt) $) coordinate (C)
($ (title.north east)+(3pt,0) $) coordinate (D)
($ (A)+(0.99*\linewidth,0) $) coordinate (E)
;
\path[fill=\mycolor!15,rounded corners=3pt]
(A)--(B)--([xshift=0.85*\linewidth]C) coordinate (E)--([xshift=0.85*\linewidth]D) coordinate (F)--cycle;
\path[fill=\mycolor!70!black,rounded corners=2pt]
(A)--(B)--([xshift=3pt]C)--([xshift=3pt]D)--cycle;
\path[left color=\mycolor,right color=\mycolor,rounded corners=3pt]
([xshift=-2pt]A)--([xshift=-2pt]B)--(C)--(D)--cycle;
\path ($ (A)!0.5!(B) +(0pt,0)$) node[anchor=west,font=\color{white}\bfseries\sffamily\selectfont]{{\faClockO\ Ví dụ~\thevd}};
\path ($ (C)!0.5!(D) +(9pt,0)$) node[anchor=west,font=\color{\mycolor}]{\taosao{#2}};
\path ($ (E)!0.5!(F) +(-9pt,0)$) node[anchor=east,font=\color{\mycolor!70!black}\itshape\bfseries]{#1};
\path ([shift={(-1pt,7pt)}]frame.south east) node[anchor=east,font=\fontsize{9.2 pt}{1pt}\selectfont\color{\mycolor!70!black}\itshape\bfseries]{#3};
},
overlay first={
\path[draw=\mycolor,rounded corners=5pt, thick] (frame.south west) coordinate (FSW)
--(frame.north west) coordinate (FNW)
-- (frame.north east) coordinate (FNE)
--(frame.south east) coordinate (FSE)
(FSW)--(FSE)
;
\path
($ (title.north west) +(-2pt,0pt)$) coordinate (A)
($ (title.south west) +(-2pt,3pt)$) coordinate (B)
($ (title.south east)+(3pt,3pt) $) coordinate (C)
($ (title.north east)+(3pt,0) $) coordinate (D)
($ (A)+(0.99*\linewidth,0) $) coordinate (E)
;
\path[fill=\mycolor!15,rounded corners=3pt]
(A)--(B)--([xshift=0.85*\linewidth]C) coordinate (E)--([xshift=0.85*\linewidth]D) coordinate (F)--cycle;
\path[fill=\mycolor!70!black,rounded corners=2pt]
(A)--(B)--([xshift=3pt]C)--([xshift=3pt]D)--cycle;
\path[left color=\mycolor,right color=\mycolor,rounded corners=3pt]
([xshift=-2pt]A)--([xshift=-2pt]B)--(C)--(D)--cycle;
\path ($ (A)!0.5!(B) +(0pt,0)$) node[anchor=west,font=\color{white}\bfseries\sffamily\selectfont]{{\faClockO\ Ví dụ~\thevd}};
\path ($ (C)!0.5!(D) +(9pt,0)$) node[anchor=west,font=\color{\mycolor}]{\taosao{#2}};
\path ($ (E)!0.5!(F) +(-9pt,0)$) node[anchor=east,font=\color{\mycolor!70!black}\itshape\bfseries]{#1};
},
overlay middle ={
frame code={yshifttext=\tcboxedtitleheight,
\path[draw=\mycolor,thick] (frame.north west) rectangle (frame.south east);
}
},
overlay last ={
\path[draw=\mycolor,rounded corners=5pt, thick] (frame.north west) coordinate (FNW)-- (frame.south west) coordinate (FSW)
--(frame.south east) coordinate (FSE)--(frame.north east) coordinate (FNE)
(FNW)--(FNE)
;
\path ([shift={(-1pt,7pt)}]frame.south east) node[anchor=east,font=\fontsize{9.2 pt}{1pt}\selectfont\color{\mycolor!70!black}\itshape\bfseries]{#3};
},
top=\tcboxedtitleheight
}
\def\vdexhead#1#2#3{\begin{vdexbox}[#1][#2][#3]}
\def\vdexend{\end{vdexbox}}
\NewDocumentEnvironment{vdex}{+!O{}O{}O{}}{
\ifblank{#1}{\def\tuychonone{}}{\def\tuychonone{#1}}
\ifblank{#2}{\def\tuychontwo{0}}{\def\tuychontwo{#2}}
\ifblank{#3}{\def\tuychonthree{}}{\def\tuychonthree{#3}}
\vdexhead{\tuychonone}{\tuychontwo}{\tuychonthree}
}{\vdexend}
\AtBeginEnvironment{vdex}{
\renewcommand{\FalseEX}{\stepcounter{dapan}{\circl{\Alph{dapan}}}}
\renewcommand{\TrueEX}{\stepcounter{dapan}{\circl{\Alph{dapan}}}}
\setcounter{numTrue}{0}%
\setcounter{numTrueSol}{0}%
%\refstepcounter{vd}
\renewcommand{\loigiai}[1]{
\end{vdexbox}
\def\vdexend{}
\begin{center}
{\color{\mycolor}\reflectbox{\Large\WritingHand}\ {\fmmfamily\LARGE Lời giải:}}
\end{center}
#1
\ifthenelse{\value{numTrueSol}>0}{
	\phantom{a}\hfill\textcolor{\mycolor!50!black}{\bfseries\sffamily \faKey~\circleTrue{\Alph{numTrueSol}}}
}{}
}
}
%%%=======================Một số lệnh mới================================%%%
%\newcommand{\bangdapanTF}[1]{
%\begin{center}
%\textbf{\textsf{BẢNG ĐÁP ÁN ĐÚNG SAI Đ/S}}
%\end{center}
%\input{Ansbook/#1}
%}

\newcommand{\ovuong}[2][2]{%
	\linepenalty100
	\exhyphenpenalty0
	~\hfill{\tikz[baseline=(char.base)]
		{%
			%	\path (-0.1,-5pt)node[anchor=east]{\reflectbox{\Large\WritingHand}\ {\sffamily Đáp số}};
			\path (0,-0.15) node (char) {};
			\draw[rounded corners=2pt] (0,-0.275) rectangle (#1,0.275);
		}%
	}
	
	\scantokens{%
		\begin{giaibt}
			#2
	\end{giaibt}}% 
}

\newcommand{\shortans}[2][3]{
	\ovuong{#2}
}

%\newcommand{\bangdapanSA}[1]{%
%\RenewEnviron{loigiaibt}[1]{
%{\small\textsf{\textbf{\color{\mycolor!50!\maunhan}\faKey BT ##1}}}\
%\tikz[baseline=(char.base)]
%{\path (0,0) node[rounded corners,draw=\mycolor!75!black,text width=2.25cm,align=center,font=\color{\maunhan}\bfseries\sffamily] (char){\BODY}%
%}%
%}%
%\begin{center}
%{\sffamily\bfseries ĐÁP SỐ PHẦN TRẢ LỜI NGẮN}
%\end{center}
%
%\medskip
%\begin{multicols}{4}%
%\noindent%
%\input{Ans/#1}
%\end{multicols}
%
%}	

%%%==================Tùy chỉnh lời giải=========================%%%

\def\luulgEXTF{
\AtBeginEnvironment{ex}{
\renewcommand{\loigiai}[1]{%
\Writetofile{ansex}{\string\def\string\writeANS{\saveans}}
\scantokens{%
\begin{giaiex}
##1
\phantom{a}\hfill{ \faKey~\writeANS}
\end{giaiex}
}% 
\Writetofile{ansbook}{\string\def\string\writeANS{\saveans}}
\scantokens{
\begin{solbook}
\writeANS
\end{solbook}%
} 
}
}
}
\newcommand{\Pointilles}[2][1.1]{%
\par\nobreak
\noindent\rule{0pt}{1.1\baselineskip}%
\foreach \i in {1,...,#2}{%
\ifnum\i=1
\noindent\makebox[\linewidth]{\rule{0pt}{#1\baselineskip}\reflectbox{\Large\WritingHand}\ {\fmmfamily\LARGE Bài làm:}\dotfill}\endgraf
\else
\noindent\makebox[\linewidth]{\rule{0pt}{#1\baselineskip}\dotfill}\endgraf
\fi
}
}

\newcommand{\DoiThanhDongKe}[1]{
\setbox0=\vbox{\hbox{
\noindent\begin{minipage}{\linewidth}%
#1 aaaaa
\end{minipage}
}}
\linepar=\ht0
\pgfmathparse{int((\linepar-\fboxsep)/(\lineheight)+1)}
\ifnum\pgfmathresult>3
\noindent%
\Pointilles{\pgfmathresult}
\else
\noindent%
\Pointilles{3}
\fi
}

\newcommand{\DoiThanhDongKeH}[1]{
\setbox0=\vbox{\hbox{
\noindent\begin{minipage}{\linewidth}%
#1 aaaaa
\end{minipage}
}}
\linepar=\ht0
\pgfmathparse{int((\linepar-\fboxsep)/(0.85\lineheight)+1)}
\ifnum\pgfmathresult>3
\noindent%
\Pointilles{\pgfmathresult}
\else
\noindent%
\Pointilles{4}
\fi
}
%%%=============================================%%%
\def\dongkeex{
\AtBeginEnvironment{ex}{
\renewcommand{\loigiai}[1]{%
\end{exbox}
\def\exend{}
\DoiThanhDongKe{##1}
}
}
}

%%%=============================================%%%
\def\dongkeHaicotex{
\AtBeginEnvironment{ex}{
\renewcommand{\loigiai}[1]{
\end{exbox}
\def\exend{}
\begin{multicols}{2}			
\DoiThanhDongKeH{##1}
\end{multicols}
}
}
}
%%%==================Tạo ô li===========================%%%
\newcommand{\DoiThanhOly}[1]{
\setbox0=\vbox{\hbox{
\noindent\begin{minipage}{\linewidth}%
#1 aaaaa
\end{minipage}
}}
\linepar=\ht0
\pgfmathparse{int((\linepar-\fboxsep)/(\lineheight)+1)}
\let\mydong\pgfmathresult	\ifnum\pgfmathresult>3
\begin{center}
\foreach \i in {0,1,...,\mydong}{%
\begin{tikzpicture}
\draw[cyan!25,ultra thin,step=0.2] (0,0) grid +(17,1);
\draw[cyan!65] (0,0) grid +(17,1);
\end{tikzpicture}\\[-1.1pt]
}
\end{center}
\else
\noindent%
\begin{center}
\begin{tikzpicture}
\draw[cyan!25,ultra thin,step=0.2] (0,0) grid +(17,3);
\draw[cyan!65] (0,0) grid +(17,3);
\end{tikzpicture}
\end{center}
\fi
}

\def\Olyex{
\AtBeginEnvironment{ex}{%
\renewcommand{\loigiai}[1]{%
\end{exbox}
\def\exend{}
\DoiThanhOly{##1}%
\Writetofile{ansbook}{\string\def\string\writeANS{\saveans}}
\scantokens{
\begin{solbook}
\writeANS
\end{solbook}
}		 
}
}
}
%%%=============================================%%%
%\def\tatloigiaiex{%
%\AtBeginEnvironment{ex}{\renewcommand{\loigiai}[1]{}}
%}
\let\oldboncot\boncot
\newboolean{TNdungsai}
\def\tatloigiaiex{%
	\AtBeginEnvironment{ex}{
		\setboolean{TNdungsai}{false}
		\gdef\loaicau{\setboolean{TNdungsai}{false}}%
		\renewcommand{\loigiai}[1]{%
			\ifthenelse{\boolean{TNdungsai}}{
				\Writetofile{ansbook}{\string\def\string\writeANS{\saveans}}
				\scantokens{
					\begin{solbook}
						\writeANS
					\end{solbook}
				}
			}{}
		}
	}
}
%%%=============================================%%%
\def\hienthiloigiaiex{%
\AtBeginEnvironment{ex}{
\renewcommand{\loigiai}[1]{%
\end{exbox}
\def\exend{}
\begin{center}
{\color{\mycolor}\reflectbox{\Large\WritingHand}\ {\fmmfamily\LARGE Lời giải:}} 
\end{center}
##1\hfill \faKey\ \circleTrue{\Alph{numTrue}} 
}
}
}

%%%=============================================%%%
\def\dongkebt{
\AtBeginEnvironment{bt}{
\renewcommand{\loigiai}[1]{
\end{btbox}
\def\btend{}
\DoiThanhDongKe{##1}
}
}
}

%%%=============================================%%%
\def\dongkeHaicotbt{
\AtBeginEnvironment{bt}{
\renewcommand{\loigiai}[1]{
\end{btbox}
\def\btend{}
\begin{multicols}{2}			
\DoiThanhDongKeH{##1}
\end{multicols}
}
}
}

%%%=============================================%%%
\def\tatloigiaibt{%
\AtBeginEnvironment{bt}{\renewcommand{\loigiai}[1]{}}
}

%%%=============================================%%%
\def\hienthiloigiaibt{%
\AtBeginEnvironment{bt}{
\renewcommand{\loigiai}[1]{%
\end{btbox}
\def\btend{}
\begin{center}
{\color{\mycolor}\reflectbox{\Large\WritingHand}\ {\fmmfamily\LARGE Hướng dẫn giải:}}
\end{center}
##1
}
}
}



%%%=============================================%%%
\def\dongkevd{
\AtBeginEnvironment{vd}{
\renewcommand{\loigiai}[1]{%
\end{vdbox}\def\vdend{}
\vspace*{-\baselineskip}
\DoiThanhDongKe{##1}
}
}%
\AtBeginEnvironment{vdex}{
\renewcommand{\loigiai}[1]{%	
\end{boxvd}%
\def\vdexend{}
\par\noindent
\DoiThanhDongKe{##1}
}
}
}

%%%=============================================%%%
\def\dongkeHaicotvd{
\AtBeginEnvironment{vd}{
\renewcommand{\loigiai}[1]{%
\end{vdbox}\def\vdend{}
\begin{multicols}{2}			
\DoiThanhDongKeH{##1}
\end{multicols}
}
}%
\AtBeginEnvironment{vdex}{
\renewcommand{\loigiai}[1]{%	
\end{vdbox}%
\def\vdexend{}
\begin{multicols}{2}			
\DoiThanhDongKeH{##1}
\end{multicols}				
}
}
}

%%%=============================================%%%
\def\tatloigiaivd{
\AtBeginEnvironment{vd}{
\renewcommand{\loigiai}[1]{}%
}%
\AtBeginEnvironment{vdex}{
\renewcommand{\loigiai}[1]{}%
}
}

\def\hienthiloigiaivd{
\AtBeginEnvironment{vd}{
\renewcommand{\loigiai}[1]{
	\end{vdbox}
	\def\vdend{}
	\begin{center}
		{\color{\mycolor}\reflectbox{\Large\WritingHand}\ {\fmmfamily\LARGE Hướng dẫn:}}
	\end{center}
	##1
	\ifthenelse{\value{numTrueSol}>0}{
	\phantom{a}\hfill\textcolor{\mycolor!50!black}{\bfseries\sffamily \faKey~\circleTrue[\mycolor!50!\maunhan]{\Alph{numTrueSol}}}
	}{}
	}
}
\AtBeginEnvironment{vdex}{
	\renewcommand{\loigiai}[1]{
	\end{vdexbox}
	\def\vdexend{}
	\begin{center}
		{\color{\mycolor}\reflectbox{\Large\WritingHand}\ {\fmmfamily\LARGE Hướng dẫn:}}
	\end{center}
	##1
		\ifthenelse{\value{numTrueSol}>0}{
		\phantom{a}\hfill\textcolor{\mycolor!50!black}{\bfseries\sffamily \faKey~\circleTrue[\mycolor!50!\maunhan]{\Alph{numTrueSol}}}
	}{}
}
}
}

%%%=============================================%%%
%%%=======================================================%%%
\newbool{Ads}
\newbool{Bds}
\newbool{Cds}
\newbool{Dds}
\makeatletter
\newcommand{\KiemtraA}{\@ifnextchar\True{\global\setbool{Ads}{true}\xdef\saveans{\saveans\string\TLdung{A}}}{\global\setbool{Ads}{false}\xdef\saveans{\saveans\string\TLsai{A}}}}
\newcommand{\KiemtraB}{\@ifnextchar\True{\global\setbool{Bds}{true}\xdef\saveans{\saveans\string\TLdung{B}}}{\global\setbool{Bds}{false}\xdef\saveans{\saveans\string\TLsai{B}}}}
\newcommand{\KiemtraC}{\@ifnextchar\True{\global\setbool{Cds}{true}\xdef\saveans{\saveans\string\TLdung{C}}}{\global\setbool{Cds}{false}\xdef\saveans{\saveans\string\TLsai{C}}}}
\newcommand{\KiemtraD}{\@ifnextchar\True{\global\setbool{Dds}{true}\xdef\saveans{\saveans\string\TLdung{D}}}{\global\setbool{Dds}{false}\xdef\saveans{\saveans\string\TLsai{D}}}}
\makeatother

\def\saveans{}
\newlength{\widthPB}
\newlength{\widthNPB}
\setlength{\widthPB}{1.1\linewidth}
\setlength{\widthNPB}{0.55\linewidth-1.1cm}
% \usepackage{longtable}
\setlength\LTpre{3pt}% khoảng cách trước longtable
\setlength\LTpost{0pt}

\newcommand{\choiceTF}[5][1]{%
	\def\saveans{}%
	\setboolean{TNdungsai}{true}%
	\gdef\loaicau{\setboolean{TNdungsai}{true}}%
	\ifthenelse{\equal{#1}{1}}{%
		\begin{enumerate}[wide=0.65cm,label*=\circlenum{\Alph*},itemsep=-3pt,topsep=-4pt]
			\item\textcolor{gray}{\rule[-3pt]{0.65cm}{0.65pt}} \KiemtraA#2\dotEX	
			\item\textcolor{gray}{\rule[-3pt]{0.65cm}{0.65pt}} \KiemtraB#3\dotEX
			\item\textcolor{gray}{\rule[-3pt]{0.65cm}{0.65pt}} \KiemtraC#4\dotEX
			\item\textcolor{gray}{\rule[-3pt]{0.65cm}{0.65pt}} \KiemtraD#5\dotEX
		\end{enumerate}
	}{
		\ifthenelse{\equal{#1}{t}}{%
			\begin{longtable}{|p{\widthPB}|C{0.35cm}|C{.35cm}|}
				\hline\rowcolor{\maunhan!8}
				\thead{\normalsize\sffamily\bfseries\fontfamily{qag}\selectfont Phát biểu}& {\bfseries\fontfamily{qag}\selectfont Đ} & {\bfseries\fontfamily{qag}\selectfont S}\\
				\hline
				\circlenum{A} \KiemtraA#2\dotEX & & \\
				\hline
				\circlenum{B} \KiemtraB#3\dotEX & & \\
				\hline
				\circlenum{C} \KiemtraC#4\dotEX & & \\
				\hline
				\circlenum{D} \KiemtraD#5\dotEX & & \\
				\hline
			\end{longtable}
		}{%
			\ifthenelse{\equal{#1}{tt}}{%
				\begin{center}
					\vspace*{-\baselineskip}
					\begin{tabular}{cc}%
						\begin{tabular}[t]{|p{\widthNPB}|C{0.35cm}|C{.35cm}|}
							\hline
							\thead{\normalsize\sffamily Phát biểu}& {\sffamily Đ} & {\sffamily S}\\
							\hline
							\circlenum{A} \KiemtraA#2\dotEX & & \\
							\hline
							\circlenum{B} \KiemtraB#3\dotEX & & \\
							\hline
						\end{tabular}~
						\begin{tabular}[t]{|p{\widthNPB}|C{0.35cm}|C{.35cm}|}
							\hline
							\thead{\normalsize\sffamily Phát biểu}&{\sffamily Đ} & {\sffamily S}\\
							\hline
							\circlenum{C} \KiemtraC#4\dotEX & & \\
							\hline
							\circlenum{D} \KiemtraD#5\dotEX & & \\
							\hline
						\end{tabular}
					\end{tabular}
				\end{center}
			}{%
				\begin{multicols}{#1}
					\begin{enumerate}[wide=0.65cm,label*=\circlenum{\Alph*},itemsep=-3pt,topsep=-4pt]
						\item\textcolor{gray}{\rule[-3pt]{0.65cm}{0.65pt}} \KiemtraA#2\dotEX	
						\item\textcolor{gray}{\rule[-3pt]{0.65cm}{0.65pt}} \KiemtraB#3\dotEX
						\item\textcolor{gray}{\rule[-3pt]{0.65cm}{0.65pt}} \KiemtraC#4\dotEX
						\item\textcolor{gray}{\rule[-3pt]{0.65cm}{0.65pt}} \KiemtraD#5\dotEX
					\end{enumerate}
				\end{multicols}	
			}
		}
	}
}


\newcommand{\LGexTF}{
	\AtBeginEnvironment{ex}{
		\renewcommand{\loigiai}[1]{%
			\par\noindent%
			{\color{dndo}\reflectbox{\Large\WritingHand}\ {\fmmfamily\LARGE Hướng dẫn.}} ##1 \hfill{{\faKey}~ 
				\ifbool{Ads}{\TLdung{A}}{\TLsai{A}}~
				\ifbool{Bds}{\TLdung{B}}{\TLsai{B}}~
				\ifbool{Cds}{\TLdung{C}}{\TLsai{C}}~
				\ifbool{Dds}{\TLdung{D}}{\TLsai{D}} 
			}%
			\Writetofile{ansbook}{\string\def\string\writeANS{\saveans}}
			\scantokens{
				\begin{solbook}
					\writeANS
				\end{solbook}
			}
		}
	}
}

\newcommand{\TLdung}[1]{%
	\tikz[baseline=(char.base)]
	{\node[shape=circle,inner sep=0.5pt,draw=orange,%fill=#1!10,
		font=\bfseries\footnotesize,text=gray,minimum size=12pt,outer sep=0pt] (char) {#1};
		\path (char) node[text=orange]{\faCheck};
	}\ignorespaces}

\newcommand{\TLsai}[1]{%
	\tikz[baseline=(char.base)]
	{\node[shape=circle,inner sep=0.5pt,draw=cyan,font=\bfseries\footnotesize,minimum size=12pt,outer sep=0pt] (char) {#1};
		\path (char) node[text=cyan,opacity=0.65]{\faTimes};
	}\ignorespaces}

\renewenvironment{Solbook}[1]{\noindent\textbf{\textsf{Câu~#1}}~}{ }
\def\itemch{\item}
\newenvironment{itemchoice}{\begin{enumerate}[wide=0.65cm,leftmargin=0.65cm,label*=\circlenum{\Alph*},itemsep=-3pt,topsep=-4pt]}{\end{enumerate}}

%%%================Lệnh bổ sung =====================%%%





