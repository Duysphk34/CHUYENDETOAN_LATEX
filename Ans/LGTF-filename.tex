\def\writeANS{\TLdung{A}\TLsai{B}\TLdung{C}\TLdung{D}}
\begin{loigiaiex}{25}
  Nguyên tử được cấu tạo từ proton, neutron và electron, và nguyên tử của cùng một nguyên tố luôn có cùng số proton. Trong nhiều nguyên tử, số proton thường nhỏ hơn số neutron.  \phantom {a}\hfill { \faKey ~\writeANS }
\end{loigiaiex}
\def\writeANS{\TLdung{A}\TLsai{B}\TLdung{C}\TLdung{D}}
\begin{loigiaiex}{26}
  Đồng vị là các nguyên tử của cùng một nguyên tố nhưng có số neutron khác nhau, có số proton giống nhau nhưng khối lượng có thể khác nhau.  \phantom {a}\hfill { \faKey ~\writeANS }
\end{loigiaiex}
\def\writeANS{\TLdung{A}\TLsai{B}\TLdung{C}\TLsai{D}}
\begin{loigiaiex}{27}
  Nguyên tử $^{12}_{6}\text {C}$ và $^{16}_{8}\text {O}$ có số neutron bằng số proton.  \phantom {a}\hfill { \faKey ~\writeANS }
\end{loigiaiex}
\def\writeANS{\TLdung{A}\TLdung{B}\TLsai{C}\TLsai{D}}
\begin{loigiaiex}{28}
  Số neutron được tính bằng cách lấy số khối (A) trừ đi số proton (Z).  \phantom {a}\hfill { \faKey ~\writeANS }
\end{loigiaiex}
\def\writeANS{\TLsai{A}\TLdung{B}\TLsai{C}\TLdung{D}}
\begin{loigiaiex}{29}
  Nguyên tử khối trung bình được tính theo tỉ lệ phần trăm của các đồng vị, và thường gần bằng số khối của nguyên tử phổ biến nhất.  \phantom {a}\hfill { \faKey ~\writeANS }
\end{loigiaiex}
\def\writeANS{\TLdung{A}\TLsai{B}\TLdung{C}\TLsai{D}}
\begin{loigiaiex}{30}
  Nguyên tố hóa học là tập hợp các nguyên tử có cùng số proton và có tính chất hóa học đặc trưng.  \phantom {a}\hfill { \faKey ~\writeANS }
\end{loigiaiex}
\def\writeANS{\TLdung{A}\TLsai{B}\TLdung{C}\TLsai{D}}
\begin{loigiaiex}{31}
  Trong một nguyên tử trung hòa, số proton bằng số electron.  \phantom {a}\hfill { \faKey ~\writeANS }
\end{loigiaiex}
\def\writeANS{\TLdung{A}\TLsai{B}\TLdung{C}\TLsai{D}}
\begin{loigiaiex}{32}
  Số khối được xác định bởi tổng số proton và neutron trong hạt nhân.  \phantom {a}\hfill { \faKey ~\writeANS }
\end{loigiaiex}
\def\writeANS{\TLdung{A}\TLsai{B}\TLsai{C}\TLdung{D}}
\begin{loigiaiex}{33}
  Nguyên tố trung hòa về điện có số electron bằng số proton, và không mang điện tích.  \phantom {a}\hfill { \faKey ~\writeANS }
\end{loigiaiex}
\def\writeANS{\TLdung{A}\TLsai{B}\TLdung{C}\TLsai{D}}
\begin{loigiaiex}{34}
  Khối lượng nguyên tử được đo bằng đơn vị khối lượng nguyên tử (u) và gần bằng khối lượng của các proton và neutron trong hạt nhân.  \phantom {a}\hfill { \faKey ~\writeANS }
\end{loigiaiex}
\def\writeANS{\TLsai{A}\TLdung{B}\TLsai{C}\TLdung{D}}
\begin{loigiaiex}{35}
  Số khối của nguyên tử là tổng số proton và neutron trong hạt nhân, và được ký hiệu bằng chữ A.  \phantom {a}\hfill { \faKey ~\writeANS }
\end{loigiaiex}
\def\writeANS{\TLsai{A}\TLdung{B}\TLdung{C}\TLsai{D}}
\begin{loigiaiex}{36}
  Nguyên tử có hạt nhân chứa các proton và neutron, trong khi các electron chuyển động xung quanh hạt nhân.  \phantom {a}\hfill { \faKey ~\writeANS }
\end{loigiaiex}
