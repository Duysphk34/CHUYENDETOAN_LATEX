\def\writeANS{\TLdung{A}\TLdung{B}\TLsai{C}\TLsai{D}}
\begin{loigiaiex}{17}
  \begin {itemchoice} \itemch Đúng. Trong một nguyên tử trung hòa tổng số proton = tổng số electron \itemch Đúng. Theo kí hiệu nguyên tố ${}^A_ZX$ trong đó $A$ là số khối, $Z$ là số hiệu nguyên tử \itemch Sai. Nguyên tử khối không có đơn vị \itemch Sai. Electron nằm ở vỏ nguyên tử, proton và neutron nằm trong hạt nhân. Nguyên tử trung hòa khi mất đi electron ở lớp vỏ ngoài cùng sẽ tạo thành ion dương (cation). \end {itemchoice}  \phantom {a}\hfill { \faKey ~\writeANS }
\end{loigiaiex}
\def\writeANS{\TLdung{A}\TLdung{B}\TLsai{C}\TLsai{D}}
\begin{loigiaiex}{18}
  \begin {itemchoice} \itemch Đúng. Số proton trong hạt nhân bằng số hiệu nguyên tử (17). \itemch Đúng. Gọi x là tỉ lệ phần trăm của ${}^{35}_{17}Cl$, ta có: 35x + 37(100-x) = 35,5 * 100. Giải ra được x = 75\%. \itemch Sai. Số neutron của đồng vị ${}^{37}_{17}Cl$ là 37 - 17 = 20. \itemch Sai. Nguyên tử khối trung bình được tính bằng trung bình có trọng số của các đồng vị, không phải trung bình cộng đơn giản. \end {itemchoice}  \phantom {a}\hfill { \faKey ~\writeANS }
\end{loigiaiex}
