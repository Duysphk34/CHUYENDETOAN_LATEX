\begin{loigiaiex}{1}
  Về orbital nguyên tử trong mô hình nguyên tử hiện đại: \begin {itemchoice} \itemch \textbf {Đúng}. Orbital là vùng không gian 3 chiều xung quanh hạt nhân, nơi có xác suất tìm thấy electron cao nhất. \itemch \textbf {Sai}. Đây là mô tả trong mô hình Bohr, không phải mô hình hiện đại. \itemch \textbf {Sai}. Lớp vỏ electron là khái niệm đơn giản hóa, không phản ánh đúng bản chất của orbital. \itemch \textbf {Sai}. Electron không có đường đi xác định trong mô hình hiện đại do tính chất sóng-hạt của nó. \end {itemchoice}  \phantom {a}\hfill { \faKey ~\circlenum {A}}
\end{loigiaiex}
\begin{loigiaiex}{2}
  Về năng lượng của electron trong mô hình Bohr: \begin {itemchoice} \itemch \textbf {Đúng}. Trong mô hình Bohr, năng lượng của electron chỉ phụ thuộc vào số lượng tử chính n. \itemch \textbf {Sai}. Số lượng tử phụ l không được đề cập trong mô hình Bohr. \itemch \textbf {Sai}. Số lượng tử từ m không được sử dụng trong mô hình Bohr. \itemch \textbf {Sai}. Spin của electron chưa được biết đến trong thời điểm Bohr đề xuất mô hình của mình. \end {itemchoice}  \phantom {a}\hfill { \faKey ~\circlenum {A}}
\end{loigiaiex}
\begin{loigiaiex}{3}
  Về nguyên lý bất định Heisenberg trong mô hình nguyên tử hiện đại: \begin {itemchoice} \itemch \textbf {Đúng}. Nguyên lý này chỉ ra giới hạn trong việc xác định đồng thời vị trí và động lượng của electron. \itemch \textbf {Sai}. Đây là quan điểm của mô hình Bohr, không phải mô hình hiện đại. \itemch \textbf {Sai}. Trong mô hình hiện đại, năng lượng của electron phụ thuộc vào nhiều yếu tố hơn. \itemch \textbf {Sai}. Xác suất tìm thấy electron không đồng đều trong toàn bộ nguyên tử. \end {itemchoice}  \phantom {a}\hfill { \faKey ~\circlenum {A}}
\end{loigiaiex}
\begin{loigiaiex}{4}
 Số lượng tử spin ms chỉ có thể nhận hai giá trị là +1/2 và -1/2. \phantom {a}\hfill { \faKey ~\circlenum {A}}
\end{loigiaiex}
\begin{loigiaiex}{5}
  Về sự chuyển dời electron trong mô hình Bohr: \begin {itemchoice} \itemch \textbf {Đúng}. Khi electron chuyển từ trạng thái năng lượng cao xuống thấp, nó phát ra một photon. \itemch \textbf {Sai}. Electron hấp thụ photon khi chuyển lên trạng thái kích thích, không phải khi về cơ bản. \itemch \textbf {Sai}. Trạng thái điện của nguyên tử không thay đổi trong quá trình này. \itemch \textbf {Sai}. Electron mất một lượng năng lượng bằng với năng lượng của photon phát ra. \end {itemchoice}  \phantom {a}\hfill { \faKey ~\circlenum {A}}
\end{loigiaiex}
\begin{loigiaiex}{6}
  Về nguyên lý Pauli trong mô hình nguyên tử hiện đại: \begin {itemchoice} \itemch \textbf {Đúng}. Nguyên lý Pauli khẳng định rằng không có hai electron nào trong một nguyên tử có thể có cùng bộ 4 số lượng tử. \itemch \textbf {Sai}. Mặc dù electron thường ghép cặp trong orbital, đây không phải là nội dung của nguyên lý Pauli. \itemch \textbf {Sai}. Mỗi orbital có thể chứa tối đa hai electron với spin ngược nhau. \itemch \textbf {Sai}. Các electron trong cùng một lớp có thể có mức năng lượng khác nhau, phụ thuộc vào orbital cụ thể. \end {itemchoice}  \phantom {a}\hfill { \faKey ~\circlenum {A}}
\end{loigiaiex}
\begin{loigiaiex}{7}
  Về bán kính quỹ đạo trong mô hình Bohr: \begin {itemchoice} \itemch \textbf {Đúng}. Bán kính quỹ đạo r tỉ lệ với $n^2$, trong đó n là số lượng tử chính. \itemch \textbf {Sai}. Mối quan hệ không phải là tỉ lệ thuận đơn giản với n. \itemch \textbf {Sai}. Bán kính không tỉ lệ với căn bậc hai của n. \itemch \textbf {Sai}. Bán kính tăng khi n tăng, không phải giảm. \end {itemchoice}  \phantom {a}\hfill { \faKey ~\circlenum {A}}
\end{loigiaiex}
\begin{loigiaiex}{8}
  Về hình dạng của orbital p trong mô hình nguyên tử hiện đại: \begin {itemchoice} \itemch \textbf {Đúng}. Orbital p có hình dạng số 8 với hai thuỳ đối xứng qua hạt nhân. \itemch \textbf {Sai}. Hình cầu là đặc trưng của orbital s, không phải p. \itemch \textbf {Sai}. Không có orbital nào có hình tròn phẳng trong mô hình hiện đại. \itemch \textbf {Sai}. Hình bông hoa bốn cánh gần giống với một số orbital d, không phải p. \end {itemchoice}  \phantom {a}\hfill { \faKey ~\circlenum {A}}
\end{loigiaiex}
\begin{loigiaiex}{9}
 Số orbital tối đa trong lớp n là $n^2$. Với lớp thứ 4, $n = 4$, nên số orbital tối đa là $4^2 = 16$. \phantom {a}\hfill { \faKey ~\circlenum {C}}
\end{loigiaiex}
\begin{loigiaiex}{10}
 Lớp M ứng với $n = 3$, chỉ có các phân lớp s, p, và d. Phân lớp f chỉ xuất hiện từ lớp N ($n = 4$) trở đi. \phantom {a}\hfill { \faKey ~\circlenum {D}}
\end{loigiaiex}
\begin{loigiaiex}{11}
 Phân lớp p có 3 orbital, mỗi orbital chứa tối đa 2 electron. Vậy số electron tối đa trong phân lớp 3p là $3 \cdot 2 = 6$. \phantom {a}\hfill { \faKey ~\circlenum {C}}
\end{loigiaiex}
\begin{loigiaiex}{12}
 Số electron tối đa trong lớp N (n = 4) là $2n^2 = 2 \cdot 4^2 = 32$. Các phát biểu khác đều sai: không có phân lớp 2d, lớp L chỉ có 2 phân lớp, và phân lớp 4f có 7 orbital. \phantom {a}\hfill { \faKey ~\circlenum {C}}
\end{loigiaiex}
\begin{loigiaiex}{13}
 Electron ở lớp ngoài cùng của nguyên tử có năng lượng cao nhất vì chúng ở xa hạt nhân nhất, chịu lực hút tĩnh điện yếu nhất từ hạt nhân và dễ dàng tham gia vào các phản ứng hóa học. \phantom {a}\hfill { \faKey ~\circlenum {D}}
\end{loigiaiex}
\begin{loigiaiex}{14}
  Orbital là khu vực không gian xung quanh hạt nhân mà tại đó xác suất có mặt electron lớn nhất.  \phantom {a}\hfill { \faKey ~\circlenum {C}}
\end{loigiaiex}
\begin{loigiaiex}{15}
  Theo nguyên lý pau-li thì mỗi một orbital có chứ tối đa 2 electron. \begin {itemize} \item Nếu AO chứa 1 e thì vẽ mũi tên hướng lên \item Nếu AO chứa 2 e thì mũi tên bên trái hướng lên và mũi tên bên phải hướng xuống. \end {itemize}  \phantom {a}\hfill { \faKey ~\circlenum {C}}
\end{loigiaiex}
\begin{loigiaiex}{16}
  Theo quy tắc hund trên một phân lớp các electron phân bố sao cho tổng số e độc thân là lớn nhất và các electron độc thân phải có chiều tự quay giống nahu  \phantom {a}\hfill { \faKey ~\circlenum {B}}
\end{loigiaiex}
\begin{loigiaiex}{17}
  Lớp thứ n có $2n^2$ electron. Lớp M ứng với n=3 có $2\cdot 3^2=18$ (electron).  \phantom {a}\hfill { \faKey ~\circlenum {D}}
\end{loigiaiex}
\begin{loigiaiex}{18}
  Nguyên tố X phân bố electron trên 3 lớp. Theo nguyên lý nững bền sau khi điền đủ số electron tối đa ở lớp 1 (2 e) , ở lớp 2 (8 e) sẽ điền tiếp 6 electron còn lại vào lớp 3. Do đó Nguyên tố X có tổng cộng $2+8+6=16$ (electron).  \phantom {a}\hfill { \faKey ~\circlenum {D}}
\end{loigiaiex}
\begin{loigiaiex}{19}
  Nguyên tố X có cấu hình electron là : $1s^22s^22p^63s^23p^5$ E lectron cuối cùng thuộc phân lớp $3p^5$ có số thứ tự lớp $n=3$ hay lớp thứ M.  \phantom {a}\hfill { \faKey ~\circlenum {C}}
\end{loigiaiex}
