\def\x{210}
\setcounter{bt}{0}
\setcounter{ex}{0}
%%%Tùy chọn 1: Kì thi(có thể bỏ trống)
%%%Tùy chọn 3: lớp (có thể bỏ trống)
%%%Tùy chọn 4: Sở/Phòng (có thể bỏ trống)
%%%Tùy chọn 5: Ngày thi (không được bỏ trống)
\begin{name}[][Hóa][10][]{Trường THCS}{2023 - 2024}
\end{name}

\Opensolutionfile{ansbt}[LOIGIAITN/KTGIUAKI1/LGTN_DE1]
\Opensolutionfile{ans}[Ans/KTGIUAKI1/DAPAN_DE1]
%%%%=============EX_1=============%%%
\begin{ex}
	Đối tượng nào sau đây là đối tượng nghiên cứu của hóa học?
	\choice{Sự quay của Trái Đất}
	{Sự sinh trưởng và phát triển của thực vật}
	{Chất và sự biến đổi về chất}
	{Tác dụng của thuốc với cơ thể người}
	\loigiai{}
\end{ex}
%%%%=============EX_2=============%%%
%\begin{ex}
%	Cho các phương pháp: lý thuyết, thực hành, vẽ hình họa, mỹ thuật. Có bao nhiêu phương pháp được sử dụng để học tập hóa học?
%	\choice{1}
%	{2}
%	{3}
%	{4}
%	\loigiai{}
%\end{ex}
%%%%=============EX_3=============%%%
%\begin{ex}
%	Ngành nào sau đây không liên quan đến hóa học?
%	\choice{Mĩ phẩm}
%	{Năng lượng}
%	{Dược phẩm}
%	{Vũ trụ}
%	\loigiai{}
%\end{ex}
%%%%=============EX_4=============%%%
%\begin{ex}
%	Trong hạt nhân nguyên tử có chứa những loại hạt nào?
%	\choice{proton, neutron}
%	{electron, neutron}
%	{electron, proton}
%	{proton, neutron, electron}
%	\loigiai{}
%\end{ex}
%%%%=============EX_5=============%%%
%\begin{ex}
%	Hạt nào sau đây mang điện tích âm?
%	\choice{Proton}
%	{Hạt nhân}
%	{Electron}
%	{Neutron}
%	\loigiai{}
%\end{ex}
%%%%=============EX_6=============%%%
%\begin{ex}
%	Khối lượng của một proton bằng
%	\choice{0,00055 amu}
%	{0,1 amu}
%	{$1 \mathrm{amu}$}
%	{0,0055 amu}
%	\loigiai{}
%\end{ex}
%%%%=============EX_7=============%%%
%\begin{ex}
%	Nguyên tố hóa học là những nguyên tử có cùng
%	\choice{số neutron}
%	{nguyên tử khối}
%	{số khổi}
%	{số proton}
%	\loigiai{}
%\end{ex}
%%%%=============EX_8=============%%%
%\begin{ex}
%	Số hiệu nguyên tử (Z) của nguyên tố hóa học không bằng giá trị nào sau đây?
%	\choice{Số hạt proton}
%	{Số hạt electron}
%	{Số điện tích dương}
%	{Số hạt neutron}
%	\loigiai{}
%\end{ex}
%%%%=============EX_9=============%%%
%\begin{ex}
%	Đồng vị là những nguyên tử có
%	\choice{cùng số proton, khác số neutron}
%	{cùng số neutron}
%	{cùng số khối}
%	{cùng số proton, cùng số neutron}
%	\loigiai{}
%\end{ex}
%%%%=============EX_10=============%%%
%\begin{ex}
%	Lớp Kcó mấy phân lớp?
%	\choice{1}
%	{3}
%	{5}
%	{7}
%	\loigiai{}
%\end{ex}
%%%%=============EX_11=============%%%
%\begin{ex}
%	Số electron tối đa trong lớp Mlà bao nhiêu?
%	\choice{2}
%	{8}
%	{32}
%	{18}
%	\loigiai{}
%\end{ex}
%%%%=============EX_12=============%%%
%\begin{ex}
%	Phân lớp nào sau đây kí hiệu sai?
%	\choice{$1 \mathrm{~s}$}
%	{$3 p$}
%	{$3 \mathrm{~d}$}
%	{$2 \mathrm{~d}$}
%	\loigiai{}
%\end{ex}
%%%%=============EX_13=============%%%
%\begin{ex}
%	Sự phóng xạ là quá trình xảy ra do yếu tố nào?
%	\choice{Sự tác động của bên ngoài}
%	{Sự tác động của con người}
%	{Sự tự phát}
%	{Do từ trường trái đất}
%	\loigiai{}
%\end{ex}
%%%%=============EX_14=============%%%
%\begin{ex}
%	Trong bảng tuần hoàn, số thứ tự của ô nguyên tố không được tính bằng
%	\choice{số proton}
%	{số electron}
%	{số hiệu nguyên tử}
%	{số khối}
%	\loigiai{}
%\end{ex}
%%%%=============EX_15=============%%%
%\begin{ex}
%	Cho biết, khối lượng của một proton bằng $1 \mathrm{amu}$ của một electron bằng $0,00055 \mathrm{amu}$ Tỉ lệ về khối lượng giữa hạt proton và hạt electron có giá trị bằng khoảng
%	\choice{181,8}
%	{1818}
%	{18,18}
%	{1,818}
%	\loigiai{}
%\end{ex}
%%%%=============EX_16=============%%%
%\begin{ex}
%	Kích thước hạt nhân so với kích thước nguyên tử bằng khoảng bao nhiêu lần?
%	\choice{$10^6$ lần}
%	{$10^7$ lần}
%	{$10^{-4}-10^{-3}$ lần}
%	{$10^{-5}-10^{-4}$ lần}
%	\loigiai{}
%\end{ex}
%%%%=============EX_17=============%%%
%\begin{ex}
%	Một nguyên tử có chứa 8 proton trong hạt nhân. Số hiệu nguyên tử của nguyên tử này là
%	\choice{8}
%	{9}
%	{16}
%	{4}
%	\loigiai{}
%\end{ex}
%%%%=============EX_18=============%%%
%\begin{ex}
%	Nguyên tử $X$ có chứa 7 proton và 8 neutron. Kí hiệu nguyên tử của $X$ là
%	\choice{$_7^8 X$}
%	{$_7^{15} X$}
%	{$_8^7 X$}
%	{$_{15}^7 X$}
%	\loigiai{}
%\end{ex}
%%%%=============EX_19=============%%%
%\begin{ex}
%	Cặp nguyên tử nào sau đây là đồng vị của nhau?
%	\choice{$_6^{12} X,_5^{10} Y$}
%	{$_1^1 M,_2^4 G$}
%	{$_8^{16} D,_8^{17} E$}
%	{$_9^{17} \mathrm{~L},_1^3 \mathrm{~T}$}
%	\loigiai{}
%\end{ex}
%%%%=============EX_20=============%%%
%\begin{ex}
%	Cho các nguyên tử với các giá trị trong bảng sau:\par
%	\begin{center}
%		\begin{tabular}{lllll}
%			\hline Nguyên tử & X& Y& G& T\\
%			\hline Tổng hạt $(p, n, e)$ & 82 & 24 & 40 & 26\\
%			\hline Số khối & 56 & 16 & 27 & 18\\
%			\hline
%		\end{tabular}\par
%	\end{center}
%	Những nguyên từ nào là đồng vị của nhau?
%	\choice{Xvà $Y$}
%	{$Y$ và $G$}
%	{Gvà T}
%	{Yvà T}
%	\loigiai{}
%\end{ex}
%%%%=============EX_21=============%%%
%\begin{ex}
%	Electron chuyển từ lớp gần hạt nhân ra lớp xa hạt nhân thì sẽ
%	\choice{thu năng lượng}
%	{giải phóng năng lượng}
%	{không thay đổi năng lượng}
%	{vừa thu vừa giải phóng năng lượng}
%	\loigiai{}
%\end{ex}
%%%%=============EX_22=============%%%
%\begin{ex}
%	Theo em, xác suất tìm thấy electron trong toàn phần không gian bên ngoài đám mây là khoảng bao nhiêu phần trăm?
%	\choice{$0 \%$}
%	{$100 \%$}
%	{khoảng $90 \%$}
%	{khoảng $50 \%$}
%	\loigiai{}
%\end{ex}
%%%%=============EX_23=============%%%
%\begin{ex}
%	Kí hiệu cấu hình electron nào sau đây viết sai?
%	\choice{$2 s^2$}
%	{$3 p^5$}
%	{$1 \mathrm{~s}^3$}
%	{$3 d^2$}
%	\loigiai{}
%\end{ex}
%%%%=============EX_24=============%%%
%\begin{ex}
%	Cấu hình electron nào sau đây là của nguyên tử Oxygen $(Z=8)$ ?
%	\choice{$1 s^2 2 s^3 2 p^3$}
%	{$1 s^2 2 s^4 2 p^2$}
%	{$1 s^2 2 s^1 2 p^5$}
%	{$1 s^2 2 s^2 2 p^4$}
%	\loigiai{}
%\end{ex}
%%%%=============EX_25=============%%%
%\begin{ex}
%	Cho các cấu hình electron sau:
%	\begin{enumerate}[(1)]
%		\begin{multicols}{2}
%			\item $1 s^2$
%			\item $1 s^2 2 s^2 2 p^3$
%			\item $1 s^2 2 s^2 2 p^6$
%			\item $1 s^2 2 s^2 2 p^6 3 s^2 3 p^1$
%			\item $1 s^2 2 s^2 2 p^6 3 s^2$
%			\item $1 s^2 2 s^2 2 p^6 3 s^2 3 p^6 4 s^1$
%		\end{multicols}
%	\end{enumerate}
%	Có bao nhiêu cấu hình electron trong các cấu hình cho trên là của nguyên tử kim loại?
%	\choice{2}
%	{3}
%	{4}
%	{5}
%	\loigiai{}
%\end{ex}
%\newpage
%%%============EX_26==============%%%
\setcounter{ex}{25}
\begin{ex}
	Cho điện tích hạt nhân $O(Z=8), \mathrm{Na}(Z=11), \mathrm{Mg}(Z=12), \mathrm{Al}(Z=13)$ và các hạt vi mô: $O^{2-}, \mathrm{Al}^{3+}, \mathrm{Al}, \mathrm{Na}, \mathrm{Mg}^{2+}, \mathrm{Mg}$. Dãy nào sau đây được xếp đúng thứ tự bán kính hạt?
	\choice{$\mathrm{Al}^{3+} < \mathrm{Mg}^{2+} < O^{2-} < \mathrm{Al} < \mathrm{Mg} < \mathrm{Na}$}
	{$\mathrm{Al}^{3+} < \mathrm{Mg}^{2+} < \mathrm{Al} < \mathrm{Mg} < \mathrm{Na} < O^{2-}$}
	{$\mathrm{Na} < \mathrm{Mg} < \mathrm{Al} < \mathrm{Al}^{3+} < \mathrm{Mg}^{2+} < O^{2-}$}
	{$\mathrm{Na} < \mathrm{Mg} < \mathrm{Mg}^{2+} < \mathrm{Al}^{3+} < \mathrm{Al} < O^{2-}$}
	\loigiai{}
\end{ex}
%%%==============EX-27=================%%%
\begin{ex}
	Cho điện tích hạt nhân $O(Z=8), \mathrm{F}(Z=9), \mathrm{Mg}(Z=12), \mathrm{Al}(Z=13),\mathrm{S}(Z=16),\mathrm{Cl}(Z=17),\mathrm{K}(Z=19),\mathrm{Ca}(Z=20)$.Hãy sắp xếp dãy các ion sau: $\mathrm{S}^{2-},\mathrm{Cl}^{-},\mathrm{K}^{+},\mathrm{Ca}^{2+},\mathrm{Al}^{3+},\mathrm{Mg}^{2+},\mathrm{O}^{2-},\mathrm{F}^{-} $ theo chiều tăng dần của bán kính?
	\choice{$\mathrm{Al}^{3+} < \mathrm{Mg}^{2+} < O^{2-} < \mathrm{Al} < \mathrm{Mg} < \mathrm{Na}$}
	{$\mathrm{Al}^{3+} < \mathrm{Mg}^{2+} < \mathrm{Al} < \mathrm{Mg} < \mathrm{Na} < O^{2-}$}
	{$\mathrm{Na} < \mathrm{Mg} < \mathrm{Al} < \mathrm{Al}^{3+} < \mathrm{Mg}^{2+} < O^{2-}$}
	{$\mathrm{Na} < \mathrm{Mg} < \mathrm{Mg}^{2+} < \mathrm{Al}^{3+} < \mathrm{Al} < O^{2-}$}
	\loigiai{}
\end{ex}
\Closesolutionfile{ans}
\Closesolutionfile{ansbt}
\label{\x}