\subsubsection{Nguyên tố hóa học}
\begin{paracol}{2}
	\Noibat{Khái niệm nguyên tố hóa học}
	\vspace{0.5cm}
	\begin{tomtat}
		\indam{Nguyên tố hoá học} là tập hợp các nguyên tử có \indam{cùng số đơn vị điện tích hạt nhân} nhưng khác nahu về số neutron. Trong nguyên tử, số đơn vị điện tích hạt nhân bằng số electron ở vỏ nguyên tử. Các electron trong nguyên tử quyết định tính chất hoá học của nguyên tử, nên các nguyên tử của cùng một nguyên tố hoá học có \indam{tính chất hoá học giống nhau}.
	\end{tomtat}
	\switchcolumn
	\begin{Bancobiet}
		Cho đến năm 2020, đã có \indam{118} nguyên tố hoá học được xác định, trong đó \indam{94} nguyên tố có nguồn gốc \indam{tự nhiên}, còn lại là nguyên tớ nhân tạo. Nguyên tố phố biến nhất ở lớp vỏ Trái Đất là oxygen $(\mathrm{O})$, chiếm khoảng $46,6 \%$ khối lượng, tiếp theo là silicon (Si) chiếm khoảng 27,7\% khối lượng.
	\end{Bancobiet}
\end{paracol}
%%%
\Noibat{Số hiệu nguyên tử, số khối, kí hiệu nguyên tử}
\begin{tomtat}
	\begin{enumerate}
		\item Số proton trong hạt nhân được gọi là \indam{số hiệu nguyên tử}.
		\begin{vidu}
			Hạt nhân nguyên tử Oxigen có 8 proton, vậy số hiệu nguyên tử của O là 8 ($Z_{O}=8$)
		\end{vidu}
		\item Tổng số proton ($Z$) và neutron ($N$) trong một hat nhân gọi là \indam{số khối}, kí hiệu là A.
		\[\hopcttoan{\mathsf{A=Z+N}}\]
		\item Kí hiệu nguyên tử $_Z^AX$ cho biết kí hiệu  hóa học của nguyên tố ($X$), số hiệu nguyên tử ($Z$) và số khối ($A$).
		\begin{center}
			\begin{tikzpicture}[declare function={d=1.5;},line cap=round,line join=round]
				\tikzstyle{stylenode} = [color=\maunhan,font=\bfseries\fontsize{25pt}{6pt}\fontfamily{qag}\selectfont,inner sep=3pt,outer sep = 3pt]
				\tikzstyle{stylenodeH} = [anchor=east,color=\maunhan!30!black,font=\bfseries\fontsize{14pt}{0pt}\fontfamily{qag}\selectfont,xshift=2pt,inner sep=3pt,outer sep = 3pt]
				\tikzstyle{stylenodeB} = [font=\small,text width =3cm,inner sep=3pt,outer sep = 3pt]
				%%%
				\path (0,0) node[stylenode] (KHHH) {Al};
				\path (KHHH.north west) node [stylenodeH] (sk) {27};
				\path (KHHH.south west) node [stylenodeH] (shnt) {13};
				\path ($(KHHH)+(d,0)$) node [stylenodeB,anchor=west] (khhh) {Kí hiệu hóa học của nguyên tố};
				\path ($(sk)+({-0.8*d},0)$) node [stylenodeB,anchor=east,align=right] (SK) {Số khối};
				\path ($(shnt)+({-0.8*d},0)$) node [stylenodeB,anchor=east,align=right] (SHNT) {Số hiệu nguyên tử};
				%%%
				\path [fill=\maunhan,rounded corners =4pt,fill opacity=.2] (sk.north west)-|(KHHH.east)|-(shnt.south west)--cycle;
				%%%
				\foreach \x/\y in {KHHH/khhh,SHNT/shnt,SK/sk}
				{\path [draw=\maunhan,thick,shorten <=-0.1cm,shorten >=-0.15cm] (\x)--(\y);}
			\end{tikzpicture}
			\captionof{figure}{Kí hiệu nguyên tử Aluminium}\label{fig:KHHHAl}
		\end{center}
	\end{enumerate}
\end{tomtat}
\subsubsection{Đồng vị, nguyên tử khối trung bình}
\Noibat{Đồng vị}\\
Các nguyên tử của cùng một nguyên tố hóa học có số neutron khác nhau  là đồng vị của nhau
\begin{vidu}[\maunhan]
	Ba loại nguyên tử Oxigen $(\mathrm{O})$ đều có cùng 8 proton trong hạt nhân nên thuộc cùng một nguyên tố hoá học, nguyên tố oxigen $(\mathrm{O})$.
	\begin{center}
		\begin{tikzpicture}[line join=round,line cap=round]
			\tikzset{%
				pics/.cd,
				hinhcau/.style args={#1}{%
					code={\path[pic actions] circle (#1 pt);}
				},
				quydao/.style args={#1}{%
					code={\path[even odd rule,pic actions] circle (#1 cm) circle 	(#1 cm-2pt);}
				}
			}
			%%%===Đồng vị 16========%%%
			\path (0,0) node (m) {\tikz{
					\foreach \p/\m in {
						(0,0)/\maunhan,(45:5pt)/\maunhan,
						(-45:5pt)/\mauphu,(90:5pt)/\mauphu,
						(0:5pt)/\mauphu,(180:5pt)/\mauphu,
						(-135:6pt)/\maunhan,(135:6pt)/\maunhan,
						(-90:7pt)/\maunhan,(90:8pt)/\mauphu,
						(0:8pt)/\maunhan,(180:8pt)/\mauphu,
						(-90:5pt)/\mauphu,(45:8pt)/\maunhan,
						(-30:10pt)/\mauphu,(135:10pt)/\maunhan
					}{
						\path \p pic[ball color=\m] {hinhcau={3}};
					}
					\path (0,0) pic[fill=\maunhan!30]{quydao={1.4}};
					\path (0,0) pic[fill=\maunhan!30]{quydao={1.8}};
					%%%Electron%%%
					%%lop1
					\foreach \g in {0,180}{
						\path (\g:{1.4cm-1pt}) pic[ball color=gray!60!white] {hinhcau={2}};
					}
					%%lop2
					\foreach \g in {45,105,165,215,275,335}{
						\path (\g:{1.8cm-1pt}) pic[ball color=gray!60!white] {hinhcau={2}};
					}
			}};
			%%%===Đồng vị 17========%%%
			\node[right=2cm of m] (h) {\tikz{
					\foreach \p/\m in {
						(0,0)/\mauphu,(45:5pt)/\maunhan,
						(-45:5pt)/\mauphu,(90:5pt)/\maunhan,
						(0:5pt)/\mauphu,(180:5pt)/\maunhan,
						(-135:6pt)/\maunhan,(135:6pt)/\mauphu,
						(-90:7pt)/\maunhan,(90:8pt)/\mauphu,
						(0:8pt)/\mauphu,(180:8pt)/\maunhan,
						(-90:5pt)/\maunhan,(45:8pt)/\mauphu,
						(-30:10pt)/\mauphu,(135:10pt)/\maunhan,
						(-135:11pt)/\mauphu
					}{
						\path \p pic[ball color=\m] {hinhcau={3}};
					}
					\path (0,0) pic[fill=\maunhan!30]{quydao={1.4}};
					\path (0,0) pic[fill=\maunhan!30]{quydao={1.8}};
					%%%Electron%%%
					%%lop1
					\foreach \g in {0,180}{
						\path (\g:{1.4cm-1pt}) pic[ball color=gray!60!white] {hinhcau={2}};
					}
					%%lop2
					\foreach \g in {45,105,165,215,275,335}{
						\path (\g:{1.8cm-1pt}) pic[ball color=gray!60!white] {hinhcau={2}};
					}
					
			}};
			%%%===Đồng vị 18========%%%
			\node [right=2cm of h] (b) {\tikz{
					\foreach \p/\m in {
						(0,0)/\mauphu,(45:5pt)/\maunhan,
						(-45:5pt)/\mauphu,(90:5pt)/\maunhan,
						(0:5pt)/\mauphu,(180:5pt)/\maunhan,
						(-135:6pt)/\maunhan,(135:6pt)/\mauphu,
						(-90:7pt)/\maunhan,(90:8pt)/\mauphu,
						(0:8pt)/\mauphu,(180:8pt)/\maunhan,
						(-90:5pt)/\maunhan,(45:8pt)/\mauphu,
						(-30:10pt)/\mauphu,(135:10pt)/\maunhan,
						(-135:11pt)/\mauphu,(90:11pt)/\mauphu
					}{
						\path \p pic[ball color=\m] {hinhcau={3}};
					}
					\path (0,0) pic[fill=\maunhan!30]{quydao={1.4}};
					\path (0,0) pic[fill=\maunhan!30]{quydao={1.8}};
					%%%Electron%%%
					%%lop1
					\foreach \g in {0,180}{
						\path (\g:{1.4cm-1pt}) pic[ball color=gray!60!white] {hinhcau={2}};
					}
					%%lop2
					\foreach \g in {45,105,165,215,275,335}{
						\path (\g:{1.8cm-1pt}) pic[ball color=gray!60!white] {hinhcau={2}};
					}
			}};
			%%%=====Ghi chú ============
			\path(h.south) node [anchor=north,yshift=-1cm] {\tikz{
					\path (0,0) pic[ball color=gray!60!white,local bounding box=e] 	{hinhcau={4}};
					\node [right= 3pt of e] (electron) {Electron};
					\path ($(electron.east)+(1cm,0)$) pic[ball color=\maunhan,local 	bounding box=p] {hinhcau={6}};
					\node [right= 3pt of p] (proton) {Proton};
					\path ($(proton.east)+(1cm,0)$) pic[ball color=\mauphu,local 	bounding box=n] {hinhcau={6}};
					\node [right= 3pt of n] (notron) {Neutron};
			}};
			%%%===Tên đồng vị===========%%%
			\path (m.south) node[anchor=north,font=\color{\maunhan}\bfseries\small\sffamily]{Đồng vị $_{\phantom{1}8}^{16}O$};
			\path (h.south) node[anchor=north,font=\color{\maunhan}\bfseries\small\sffamily]{Đồng vị $_{\phantom{1}8}^{17}O$};
			\path (b.south) node[anchor=north,font=\color{\maunhan}\bfseries\small\sffamily]{Đồng vị $_{\phantom{1}8}^{18}O$};
		\end{tikzpicture}
		\captionof{figure}{Ba đồng vị phổ biến của Oxigen}
		\label{fig:dongvioxi}
	\end{center}
\end{vidu}
\Noibat{Nguyên tử khối trung bình}
\Noibat[][][\faArrowCircleOLeft]{Nguyên tử khối}

Nguyên tử khối là \indam{khối lượng tương đối} của một nguyên tử, cho biết nguyên tử đó nặng gấp bao nhiêu lần 1 \indam{amu}\footnote{Viết tắt của cụm từ \lq\lq atom mass unit \rq\rq}.

\Noibat[][][\faArrowCircleOLeft]{Nguyên tử khối trung bình}

\indam{Công thức tính nguyên tử khối trung bình:}
\hopcttoan{\overset{\resizebox{0.27cm}{!}{\_}}{\mathsf{A}}=\frac{\mathsf{X} \cdot \mathsf{x}+\mathsf{Y} \cdot \mathsf{y}+\mathsf{Z} \cdot \mathsf{z}+\ldots}{\mathsf{x}+\mathsf{y}+\mathsf{Z}}}\\
Trong đó :
\begin{itemize}
	\item X, Y, Z, $\ldots$ lần lượt là số khối của các đồng vị
	\item x, y, z ,$\ldots$ là phần trăm số nguyên tử các đồng vị tương ứng.
\end{itemize}
\begin{tongket}{Em đã học}
	\begin{itemize}
		\item  Nguyên tố hoá học là tập hợp các nguyên tử có cùng số đơn vị điện tích hạt nhân.
		\item  Đồng vị là những nguyên tử có cùng số đơn vị điện tích hạt nhân (cùng số proton) nhưng có số neutron khác nhau.
		\item  Kí hiệu của nguyên tử: $\mathrm{Z} X$.
		\item  Nguyên tử khối cho biết khối lượng nguyên tử đó nặng gấp bao nhiêu lần đơn vị khối lượng nguyên tử.
		\item  Nguyên tử khối của một nguyên tố là nguyên tử khối trung bình của hỗn hợp các đồng vị của nguyên tố đó.
	\end{itemize}
\end{tongket}